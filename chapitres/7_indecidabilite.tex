% !TEX root = ../cours.tex
\chapter{Indécidabilité}

Dans ce chapitre, nous allons étudier certains problèmes qui sont \og \textit{indécidables} \fg{}, c'est-à-dire des problèmes qui ne sont pas décidés par une machine de Turing.
Pour construire un premier problème indécidable, nous allons à nouveau recourir à la technique de la \textit{diagonalisation}.

\section{Existence d'un langage hors de \textit{RE}}

Rappelons nous que l'ensemble des mots dans $\Sigma^*$ est dénombrable: il existe une bijection $f$ entre $\mathbb{N}$ et $\Sigma^*$.
Notons $f^{-1}$ l'inverse.
De même, l'ensemble des machines de Turing sur un certain alphabet est aussi dénombrable.
Notons $g$ la bijection entre $\mathbb{N}$ et l'ensemble des machines de Turing et $g^{-1}$ son inverse.
Considérons le langage $L$ définit comme:
\[
L_0 \eqdef \{\ w \in \Sigma^*\ |\ g(f^{-1}(w))\text{ n'accepte pas }w\ \}
\]

Admettons, vers une contradiction, que $L_0$ soit dans la classe \textit{RE}.
Il existe donc une machine de Turing $M$ telle que $L(M) = L_0$.
Examinons maintenant l'index $i$ de la machine de Turing $M$ dans la bijection $g$:
\[
i \eqdef g^{-1}(M)
\]
Regardons si $f(i)$, un mot de $\Sigma^*$, appartient à $L_0$.
Considérons les deux cas, soit $f(i) \in L_0$, soit $f(i) \not\in L_0$.
\begin{enumerate}
\item Dans la cas où $f(i) \in L_0$, alors par définition:
\[
g(f^{-1}(f(i))\text{ n'accepte pas }f(i)
\]
Or, $g(f^{-1}(f(i)) = g(i) = M$, et donc $M$ n'accepte pas $f(i)$. On a donc $f(i) \not\in L(M) = L_0$, et donc une contradiction.
\item
Dans le cas où $f(i) \not\in L_0$, alors on a que $f(i) \not\in L(M)$ et donc que $M$ n'accepte pas $f(i)$.
De plus, observons que $M = g(i)$, et $i = f^{-1}(f(i))$. On a donc:
\[
g(f^{-1}(f(i))) \text{ n'accepte pas } f(i)
\]
Par définition, on a donc que:
\[
f(i) \in L_0
\]
Comme $L_0 = L(M)$, on arrive à une contradiction. 
\end{enumerate}

\section{Existence d'un langage dans \textit{RE} mais hors de \textit{co-RE}}

Examinons le complément de $L_0$, noté $\overline{L_0}$:
\[
\overline{L_0} = \{\ w \in \Sigma^*\ |\ g(f^{-1}(w))\text{ accepte }w\ \}
\]
Ce langage fait partie de \textit{RE}: En effet, il est possible de concevoir une machine qui accepte ce langage.
La façon de procédé est la suivante:
\begin{itemize}
\item Étant donné un mot d'entrée $w$, la machine énumère les mots de $\Sigma^*$ jusqu'à trouver l'index $i$ qui correspond à $w$.
\item La machine énumère les machines de Turing jusqu'à la machine $M$ d'index $i$.
\item La machine simule l'exécution de $M$ sur $w$.
\end{itemize}
Le langage $\overline{L_0}$ est donc dans \textit{RE}.
Par contre, il n'est pas dans \textit{co-RE}.
En effet, $L_0$ n'est pas dans \textit{RE}.
On a donc un langage, $\overline{L_0}$, qui est dans \textit{RE} mais pas dans $\textit{R} = \textit{RE} \cap \textit{co-RE}$.
Le langage est \textit{récursivement énumérable}, mais pas \textit{récursif}.
Le problème est dit \textit{indécidable}: il n'existe pas de machine de Turing qui \textit{décide} le langage.

\section{Problème de l'arrêt}

Le problème de l'arrêt consiste à savoir si une machine de Turing s'arrête sur une entrée donnée ou boucle à l'infini. le langage correspond au problème est:
\[
L_\textsc{Halt} \eqdef \{\ <M, w>\ |\ M\text{ s'arrête sur }w\ \}
\]
Comme on va le montrer, le problème de l'arrêt est indécidable.
Pour ce faire, on va procéder par \textit{réduction}: l'idée est de partir de l'hypothèse que le langage est décidable, et de montrer une contradiction en construisant une machine de Turing qui décide un problème indécidable.
Dans le cas présent, on va montrer que s'il existe une machine de Turing $M_\textsc{Halt}$ qui décide de problème de l'arrêt, alors on peut construire une machine de Turing $M_0$ qui décide le langage (indécidable) $\overline{L_0}$.

La construction de la machine de Turing $M_0$ est simple. Il s'agit simplement d'une adaptation de la machine précédemment introduite qui accepte $\overline{L_0}$:
\begin{itemize}
\item Étant donné un mot d'entrée $w$, la machine énumère les mots de $\Sigma^*$ jusqu'à trouver l'index $i$ qui correspond à $w$.
\item La machine énumère les machines de Turing jusqu'à la machine $M$ d'index $i$.
\item La machine utilise $M_\textsc{HALT}$ pour déterminer si $M$ s'arrête ou pas sur $w$. Dans le cas où $M$ bouclerait sur $w$, alors la machine s'arrête immédiatement et rejette le mot.
\item La machine simule l'exécution de $M$ sur $w$. L'exécution est garantie à se point de terminer, car $M_\textsc{HALT}$ accepte $<M, w>$.
\end{itemize}

L'existence d'une telle machine $M_0$ est en contradiction avec le fait que $L_0$ est indécidable.
On en conclut donc que l'hypothèse que le problème de l'arrêt est décidable est fausse.
Le problème de l'arrêt est donc indécidable.

\section{Langage universel}

On appelle le langage universel $LU$ l'ensemble des paires $<M, w>$ telles que $M$ est une machine de Turing qui accepte $w$.
\[
LU \eqdef \{\ <M, w>\ |\ M \text{ accepte } w\ \}
\]
Le langage $LU$ est indécidable.
Il n'existe pas de machine de Turing, qui, étant donné une machine $M$ est un mot $w$, décide si $M$ accepte $w$.
L'indécidabilité de $LU$ peut être montrée simplement par réduction vers $\overline{L_0}$ ou vers le problème de l'arrêt, par exemple. 