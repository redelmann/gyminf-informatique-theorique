% !TEX root = ../cours.tex
\chapter{Complexité}

Dans ce dernier chapitre, nous allons nous intéresser à la notion de \og \textit{complexité} \fg{}.
L'étude de la complexité d'un algorithme cherche à calculer l'utilisation en ressources d'un algorithme ou d'une machine de Turing en fonction de son entrée.
La ressource en question sera généralement le temps.
Dans ce cas, on parle de \textit{complexité temporelle}.
Il est aussi possible d'étudier d'autres métriques que le temps.
Par exemple, la quantité de mémoire requise est souvent intéressante à analyser.
On parle dans ce cas là de \textit{complexité en espace}.
Dans ce chapitre, nous nous concentrerons sur la complexité temporelle.

\section{Notation $O$}

Pour parler de \textit{complexité}, nous allons utiliser la notation $O$ afin d'étudier le comportement asymptotique des fonctions. Comme on le vera par la suite, on modélisera la complexité d'un algorithme comme une fonction de la taille de l'entrée de l'algorithme.

L'ensemble $O(g)$ d'une fonction $g: \mathbb{N} \to \mathbb{N}$ est l'ensemble de toutes les fonctions $f$ telles que $f$ est, à partir d'un certain point, bornée par un multiple de la fonction $g$.

\[
O(g) \eqdef \{\ f\ |\ \exists n_0. \exists c. \forall n.\ n > n_0 \implies f(n) \leq c \cdot g(n)\ \}
\]

Autrement dit, on a que:
\[
f \in O(g) \iff \exists n_0. \exists c. \forall n.\ n > n_0 \implies f(n) \leq c \cdot g(n)
\]

À partir d'un certain point, la fonction $f$ est bornée par la fonction $g$ (multipliée par une constante).

\section{Complexité dans le pire des cas}

La complexité d'une machine de Turing correspond au temps d'exécution de la machine, c'est-à-dire que la complexité correspond au nombre de transitions prises par la machine durant l'exécution.
Évidement, ce nombre dépends généralement de l'entrée.

Décrire exactement le temps d'exécution d'une machine en fonction de l'entrée s'avèrerait trop complexe.
Par simplification, on calcul la complexité en fonction de la \emph{la taille} de l'entrée, et non du contenu exact de l'entrée.
Comme plusieurs entrées partagent la même taille, on considèrera le \emph{pire des cas}.

Au lieu de calculer la complexité de manière exact, on utilisera généralement la notation $O$.
Cette notation permet de s'affranchir de détails d'implémentation.

\section{Classe de complexité $P$}

La classe de complexité $P$ est l'ensemble des langages qui sont décidés par une machine de Turing déterministe dont la complexité dans le pire des cas est bornée par un polynôme.

Intuitivement, la classe $P$ correspond approximativement au problèmes qui sont décidables de façon \emph{efficace}.

\section{Classe de complexité $NP$}

La classe de complexité $NP$ est l'ensemble des langages qui sont décidés par une machine de Turing \textbf{non-déterministe} dont la complexité dans le pire des cas est bornée par un polynôme.

C'est aussi la classe des problèmes où la \textbf{vérification} d'une solution (elle-même de taille polynomiale) au problème par une machine de Turing déterministe prend un temps polynomial.

Déterminer si $P = NP$ est un problème ouvert.
La plupart des chercheurs pensent que $P \neq NP$, mais aucune preuve à ce jour n'existe.

\section{Problèmes $NP$-complets}

Parmi les problèmes $NP$, on considère un sous-ensemble des problèmes \textit{les plus durs}, qu'on appelle problèmes $NP$-complets.
Un problème est $NP$-complet si et seulement si:
\begin{enumerate}
\item Le problème est dans $NP$, et
\item Il est possible de résoudre tout problème dans $NP$ en temps polynomial sur un machine de Turing déterministe étant donné un oracle pour le premier problème.
\end{enumerate} 

\subsection{Théorème de Cook}

Le théorème de Cook énonce que le problème \texttt{SAT} est $NP$-complet.
Le langage du problème \texttt{SAT} consiste en l'ensemble des formules propositionnelles satisfiables.
En général, on restreint le domaine aux formules en forme normale conjonctive: La formules est une conjonctions de clauses, et chaque clause est une disjonctions de littéraux. Chaque littéral est soit une variable, soit la négation d'une variable.

La preuve du théorème montre comment, étant donné une machine de Turing non-déterministe qui s'exécute en un temps polynomial, construire une formule propositionnelle (en temps polynomial) qui soit satisfiable si et seulement si la machine de Turing a une exécution acceptante.

Le théorème de Cook établi l'existence d'un premier problème $NP$-complet.

\subsection{Réduction polynomiales}

Une fois qu'on a établi qu'un problème $A$ est dans NP, pour démontrer que $A$ est NP-complet l'on procède généralement par \og \textit{réduction polynomiale} \fg{}:
On montre qu'il est possible, étant donné un oracle pour le problème $A$, de décider un problème $NP$-complet $B$ en temps polynomial déterministe.

Pour ce faire, on décrit une procédure effective qui prend pour entrée une instance de $B$ et la transforme (en temps polynomial) en instance de $A$.
À ce moment là, l'oracle pour $A$ est appelé et sa réponse est utilisée pour déterminer le résultat.

On établi donc que si $A$ est dans $P$, alors le problème $B$ est lui aussi dans $P$, et par conséquent tous les problèmes dans $NP$ sont dans $P$, et donc $P = NP$. Il suffit donc de trouver un seul algorithme polynomial qui décide d'un seul problème NP-complet afin de montrer que $P = NP$.
Le fait qu'un tel algorithme n'aie pas encore été trouvé fait lourdement penser que $P \neq NP$.
