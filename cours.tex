\documentclass[12pt,french,a4paper]{memoir}
\usepackage{ae,lmodern}
\usepackage[francais]{babel}
\usepackage[utf8]{inputenc}
\usepackage[T1]{fontenc}
\usepackage{amsmath}
\usepackage{amssymb}
\usepackage{mathtools}
\usepackage{proof}
\usepackage[hidelinks]{hyperref}
\usepackage[french]{cleveref}
\usepackage{multirow}
\usepackage{tikz}

% Définition de la commande pour le signe = avec "déf" aussi dessus.
\newcommand\eqdef{\mathrel{\overset{\makebox[0pt]{\mbox{\normalfont\tiny\sffamily déf}}}{=}}}

\begin{document}

\title{Fondamentaux formels\\/\\Informatique théorique}
\author{Dr Romain Edelmann\\EPFL}
\date{2021}
\maketitle

\frontmatter

\chapter*{Remerciements}

Un grand merci à Patrick Rossi, Murièle Jacquier, James Jane, Sebastien Murphy et Daniel Kessler pour leur nombreuses contributions à ce document, aux exercices et à leur solutions.

\newpage

\tableofcontents

\mainmatter

\input{chapitres/1_calcul_des_propositions.tex}

\input{chapitres/2_calcul_des_predicats.tex}

\input{chapitres/3_ensembles_et_langages.tex}

\input{chapitres/4_langages_reguliers.tex}

% !TEX root = ../cours.tex
\chapter{Langages hors contexte}

\section{Automates à pile}

Un \og \textit{automate à pile non-déterministe} \fg, ou simplement \og \textit{automate à pile} \fg, est une extension aux automates finis non-déterministes qui leur ajoute une mémoire non-bornée sous forme de \textit{pile}.
Une pile est une structure de donnée qui représente une séquence d'éléments accessible de manière \textit{last-in/first-out} (LIFO): les derniers éléments ajoutés sont les premiers à être retirés.
À chaque transition, l'automate peut consulter le sommet de cette pile et le replacer par une autre séquence de symbole.

Formellement, un automate à pile est un $n$-tuplet:
\[
(Q, \Sigma, \Gamma, \Delta, Z, s, F)
\]
Avec pour composants:
\begin{enumerate}
\item
Un ensemble d'états $Q$.
\item
Un alphabet d'entrée $\Sigma$.
\item
Un alphabet de pile $\Gamma$.
\item
Une relation de transition $\Delta \subseteq (Q \times \Sigma^* \times \Gamma^*) \times (Q \times \Gamma^*)$.
\item
Un symbole initial de pile $Z \in \Gamma$.
\item
Un état initial $s \in Q$.
\item
Un ensemble d'états finaux $F \subseteq Q$.
\end{enumerate}

\subsection{Configuration}

Une \og \textit{configuration} \fg{} d'un automate à pile contient l'état de l'automate, le reste du mot d'entrée, ainsi que le contenu de la pile.
Ainsi, les configurations d'un automate à pile $(Q, \Sigma, \Gamma, \Delta, Z, s, F)$ sont des éléments de l'ensemble:
\[
Q \times \Sigma^* \times \Gamma^*
\]

\subsection{Dérivabilité}

Étant donné un automate à pile $P = (Q, \Sigma, \Gamma, \Delta, Z, s, F)$, on dit qu'une configuration $(q', w_2, p' \cdot r)$ est \og \textit{dérivable en une étape} \fg{} d'une configuration $(q, w_1 \cdot w_2, p \cdot r)$ si et seulement si:
\[
((q, w_1, p), (q', p')) \in \Delta
\]
Intuitivement, la transition se fait de l'état $q$ à l'état $q'$, consomme en entrée le mot $w_1$, et remplace la séquence $p$ au sommet de la pile par la séquence~$p'$.

Étant donnés un automate à pile $P$ et deux configurations $(q, w, p)$ et $(q', w', p')$ de cet automate, on note $(q, w, p) \vdash_P (q', w', p')$ le fait que $(q', w', p')$ est dérivable en une étape de $(q, w, p)$.

Une configuration $(q', w', p')$ est \og \textit{dérivable} \fg{} d'une configuration $(q, w, p)$ si et seulement si il existe un nombre $n > 1$ de configurations $(q_i, w_i, p_i)$ pour $i$ allant de $1$ à $n$ telles que:
\begin{enumerate}
\item $q_1 = q, w_1 = w, p_1 = p$
\item $q_n = q', w_n = w', p_n = p'$
\item $(q_i, w_i, p_i) \vdash_P (q_{i+1}, w_{i+1}, p_{i+1})$ pour tout $i < n$.
\end{enumerate}
Étant données deux configurations $(q, w, p)$ et $(q', w', p')$, on note $(q, w, p) \vdash_P^* (q', w', p')$ le fait que $(q', w', p')$ est dérivable de $(q, w, p)$.

\subsection{Acceptation}

Étant donnés un automate à pile $P = (Q, \Sigma, \Gamma, \Delta, Z, s, F)$ et un mot $w$, on dit que $P$ \og \textit{accepte} \fg{} $w$ si et seulement si il existe une pile $p \in \Gamma^*$ et un état final $f \in F$ tels que $(s, w, Z) \vdash_P^* (f, \epsilon, p)$.
Le langage d'un automate à pile est l'ensemble des mots qu'il accepte.

On considère qu'un automate à pile accepte un mot s'il est possible de le traiter en entier et de terminer dans un état acceptant, peut importe le contenu de la pile dans la dernière configuration.
On parle d'automate \og \textit{sur état final} \fg{}. Une définition alternative, et d'expressivité équivalente, consiste à considérer une configuration comme finale lorsque à la fois le mot d'entrée et la pile sont vides, et ce peu importe l'état (acceptant ou non).
On parle dans ce cas d'automate \og \textit{sur pile vide} \fg{}.

\subsection{Théorème du gonflement}

Étant donné un langage $L$ décrit par un automate à pile $P$, il existe un nombre $p$ tel que tout mot $w$ de $L$ de taille supérieure ou égale à $p$ peut être décomposé en 5 sous-parties $x, u, y, v, z$ telles que:
\begin{enumerate}
\item $w = x \cdot u \cdot y \cdot v \cdot z$,
\item $|u \cdot v| > 0$
\item $|u \cdot y \cdot v| \leq p$
\item $\forall i, x \cdot u^i \cdot y \cdot v^i \cdot z \in L$. 
\end{enumerate}

\subsection{Automates à pile déterministe}

Les automates à pile tels que nous les avons décrit sont non-déterministes.
Il existe cependant une version déterministe de ce formalisme.

Pour définir la notion d'automate à pile déterministe, on introduit la notion de \og \textit{transitions compatibles} \fg{}.
Étant donné deux transitions $t_1, t_2 \in \Delta$, on dit que les transitions sont compatibles si et seulement si il existe une configuration dans laquelle les deux transitions sont possibles.
Quand un automate à pile n'a pas de transitions compatibles, alors on parle d'un automate à pile déterministe.

À noter, et peut-être de façon surprenante, les automates à pile déterministes sont moins expressifs que les automates non-déterministes:
il existe des langages décrits par des automates à pile non-déterministes qui ne
sont pas descriptibles par un automate à pile déterministe.

\section{Grammaires non-contextuelles}

Une \og \textit{grammaire non-contextuelle} \fg{} $G$ est composée de:
\begin{enumerate}
\item Un ensemble fini de symboles $V$,
\item Un alphabet $\Sigma \subseteq V$ de symboles \og \textit{terminaux} \fg{},
\item Un ensemble fini de règles $R \subseteq (V - \Sigma) \times V^*$,
\item Un symbole initial $S \in (V - \Sigma)$.
\end{enumerate}
On note $G = (V, \Sigma, R, S)$.
On appelle les éléments de $(V - \Sigma)$ des symboles \og \textit{non-terminaux} \fg{}.
On utilise généralement des lettres majuscules pour dénoter les non-terminaux et des lettres minuscules pour les terminaux.
Chaque règle d'une grammaire non-contextuelle met en relation un symbole non-terminal de la grammaire avec une séquence finie de symboles, qu'ils soient terminaux ou non-terminaux. On note généralement $A \to w$ la règle $(A, w) \in R$.

\subsection{Dérivation}

Étant donné une séquence non-vide de symboles $s \in V^+$ et une séquence de symboles $s' \in V^*$, une grammaire non-contextuelle $G = (V, \Sigma, R, S)$ permet de \og \textit{dériver en une étape} \fg{} $s'$ de $s$ si et seulement si:
\begin{enumerate}
\item $s = s_1 \cdot A \cdot s_2$ pour certains $A \in V - \Sigma$ et $s_1, s_2 \in V^*$,
\item $s' = s_1 \cdot w \cdot s_2$ pour un mot $w \in V^*$, et
\item la grammaire $G$ contient règle $A \to w$.
\end{enumerate}
Dans ce cas, on note:
\[
s \Rightarrow_G s'
\]
On dit qu'une grammaire $G$ permet de \og \textit{dériver} \fg{} une séquence de symbole $s' \in V^*$ d'une séquence $s \in V^*$ si et seulement si il existe un nombre $n \geq 1$ et $n$ mots $w_i$ pour $i$ de $1$ à $n$ tels que:
\begin{enumerate}
\item $w_1 = s$,
\item $w_n = s'$,
\item $w_i \Rightarrow_G w_{i+1}$ pour chaque $i < n$.
\end{enumerate}
Dans ce cas, on note:
\[
s \Rightarrow_G^* s'
\]
Une grammaire permet de dériver un mot $w \in \Sigma^*$ si et seulement si:
\[
S \Rightarrow_G^* w
\]
On appelle la séquence des $w_1 \dots w_n$ une \og \textit{dérivation} \fg{} de $w$.
Une dérivation dans laquelle le symbole non-terminal le plus à gauche de chaque séquence est remplacé à chaque étape est appelé une \og \textit{dérivation à gauche} \fg.

\subsection{Langage d'une grammaire}

On note $L(G)$ le langage d'une grammaire $G = (V, \Sigma, R, S)$.
\[
L(G) \eqdef \{\ w \in \Sigma^*\ |\ S \Rightarrow_G^* w\ \}
\]

\subsection{Équivalence grammaires non-contextuelles / automates à pile}

Les grammaires non-contextuelles décrivent exactement les mêmes langages que les automates à pile:
Pour chaque langage généré par une grammaire, il existe un automate à pile qui décrit le langage, et inversement.

\section{Arbres d'analyse}

Un arbre d'analyse est un arbre où chaque noeud intérieur est annoté d'un symbole non-terminal et chaque feuille de l'arbre est annotée d'un symbole terminal ou de $\epsilon$.
Pour chaque noeud intérieur annoté d'un symbole non-terminal $A$, les noeuds descendants directs sont chacun annoté d'un symbole $s_i \in V$ tels que la grammaire comprend la règle $A \to s_1 \cdot \ldots \cdot s_n$.
Chaque feuille $\epsilon$ de l'arbre d'analyse est l'unique descendant de leur parent. Dans ce cas, la grammaire comprend la règle $A \to \epsilon$.

La concaténation des annotations des feuilles d'un arbre d'analyse résulte toujours en un mot généré par la grammaire.
De plus, pour chaque mot généré par la grammaire, il existe au moins un arbre d'analyse.

On dit qu'un mot $w$ est \og \textit{ambigu} \fg{} dans une grammaire s'il existe au moins deux arbres d'analyse distincts.
Une grammaire $G$ est dite \og \textit{ambiguë} \fg{} si au moins un de ses mots est ambigu.

\section{Forme normale de Chomsky}

Une grammaire $G = (V, \Sigma, R, S)$ est dite sous \og \textit{forme normale de Chomsky} \fg{} si et seulement si chaque règle de $R$ a la forme:
\begin{itemize}
\item $A \to B \cdot C$ pour des non-terminaux $A$, $B$ et $C$,
\item $A \to a$ pour un non-terminal $A$ et un terminal $a$, ou
\item $S \to \epsilon$.
\end{itemize}
Si la grammaire a pour règle $S \to \epsilon$, alors $S$ ne peut pas apparaitre dans la partie droite d'une règle.

Chaque grammaire non-contextuelle peut être convertie en une grammaire équivalente sous forme normale de Chomsky.
Le processus de conversion opère en 5 étapes:
\begin{enumerate}
\item Lors de la première étape, un nouveau symbole initial $S_0$, est introduit, ainsi que la règle $S_0 \to S$.
\item Pour chaque symbole terminal $a$, on introduit un nouveau symbole non-terminal $N_a$, ainsi que la règle $N_a \to a$.
Chaque autre règle remplace les occurrences des terminaux $a$ par le non-terminal $N_a$.
\item Chaque règle $A \to B_1 \cdot \ldots \cdot B_n$ avec $n > 2$ est remplacé par de nouvelles règles avec des parties droites de taille $2$:
\begin{align*}
A &\to B_1 \cdot Z_1\\
Z_1 &\to B_2 \cdot Z_2\\
&\hspace{2em} \vdots\\
Z_{n - 3} &\to B_{n - 2} \cdot Z_{n - 2}\\
Z_{n - 2} &\to B_{n - 1} \cdot B_n
\end{align*}
\item En quatrième temps, toutes les règles de la forme $A \to \epsilon$ pour $A$ différent du symbole initial sont éliminées.
Pour ce faire, les non-terminaux \og \textit{annulables} \fg{} sont calculés: Un terminal est annulable s'il peut générer le mot vide $\epsilon$.
Ensuite, pour chaque règle $A \to B \cdot C$ qui contient un symbole annulable $B$, une nouvelle règle $A \to C$ est ajoutée.
De même dans le cas où $C$ serait annulable: dans ce cas, la règle $A \to B$ est ajoutée.
Si le symbole $S_0$ initial est annulable, alors la règle $S_0 \to \epsilon$ est ajoutée.
Toutes les autres règles de la forme $A \to \epsilon$ sont retirées.
\item
Finalement, dans un cinquième temps, les règles dont la partie de droite est un unique non-terminal sont supprimées.
Ces règles sont appelées règles \og \textit{unitaires} \fg{}.
Étant donné une règle unitaire $A \to B$, on ajoute pour chaque règle $B \to s$ une règle $A \to s$. Une fois effectué, on peut supprimer la règle $A \to B$. On n'ajoute cependant pas une règle si elle a été préalablement supprimée dans le courant de l'exécution de cette étape, comme cela pourrait se produire en cas de cycles. 	 
\end{enumerate}

\section{Algorithme de Cocke-Younger-Kasami}

L'algorithme de Cocke-Younger-Kasami (CYK) est un algorithme d'analyse syntaxique opérant sur les grammaires non-contextuelles sous forme normale de Chomsky. L'algorithme fait usage d'une technique algorithmique appelée \textit{programmation dynamique}.

\section{Types de grammaires}

Les grammaires que nous avons considérées dans ce chapitre sont appelées non-contextuelles:
la partie gauche des règles est uniquement composée d'un seul non-terminal.
Le symbole n'a pas de \textit{contexte}.
On appelle les grammaires non-contextuelles des grammaires de \textbf{type 2}.
Il existe cependant d'autres types de grammaires:
\begin{itemize}
\item Les grammaires de \textbf{type 0} n'ont pas de restriction sur les règles. La partie de gauche des règles peut comprendre des symboles arbitraires.
\item Les grammaires de \textbf{type 1} ont pour restriction que la partie de gauche soit de taille plus petite ou égale à la partie de droite.
On autorise cependant une règle à enfreindre cette contrainte: la règle $S \to \epsilon$ pour le symbole initial $S$, et ce à condition que $S$ n'apparaisse pas dans la partie droite d'une règle.
\item Les grammaires de \textbf{type 2} sont traitées dans ce chapitre.
\item les grammaires de \textbf{type 3} sont appelées \og \textit{grammaires régulières} \fg{}. Chaque règle d'une telle grammaire est de la forme $A \to wB$ ou $A \to w$ pour des non-terminaux $A$ et $B$, et un mot $w \in \Sigma^*$. Les grammaires régulières génèrent exactement les langages réguliers.
\end{itemize}
L'expressivité des différents types de grammaires est strictement décroissante:
il existe des langages qui sont décrit par des grammaires de type 0, mais pas de type 1, 2, ou 3.
De même, il existe des langages décrits par des grammaires de type 1, mais pas de type 2 ou 3, et ainsi de suite.
Par contre, pour les langages décrits par une grammaire de type $n > 0$, il existe forcément une grammaire de type $n - 1$ qui décrit ce langage.
À noter qu'il s'agit parfois simplement de la même grammaire.

% !TEX root = ../cours.tex
\chapter{Langages récursifs}

\section{Machines de Turing déterministes}

Une \og \textit{machine de Turing déterministe} \fg{} est un automate fini avec un ruban de taille infinie que l'automate peut parcourir, lire et modifier.
Le ruban contient initialement le mot d'entrée, suivit d'une infinité de \textit{symboles blancs} qui dénotent les cases vides du ruban.
Une tête de lecture/écriture permet de lire et de modifier le contenu du ruban.
Initialement, cette tête se situe sur le premier symbole du ruban, qui contient le premier symbole du mot d'entrée (si le mot n'est pas vide).
À chaque transition, en fonction de l'état courant et du symbole pointé par la tête de lecture, l'automate détermine le nouveau état de la machine et remplace le symbole lu sur le ruban par un autre symbole.
En plus de cela, l'automate bouge la tête de lecture d'une case, et ce soit à droite de la case courante, soit à gauche.

Plus formellement, une machine de Turing déterministe est un élément un $7$-uplet $(Q, \Gamma, \Sigma, \delta, s, B, F)$ tel que:
\begin{enumerate}
\item
$Q$ est un ensemble fini d'états,
\item
$\Gamma$ est l'alphabet de ruban,
\item
$\Sigma \subseteq \Gamma$ est l'alphabet d'entrée,
\item
$\delta : ((Q - F) \times \Gamma) \not\to (Q \times \Gamma \times \{L, R\})$ est la fonction de transition. Le symbole $R$ indique un déplacement de la tête de lecture/écriture sur la droite du ruban, et $L$ un déplacement sur la gauche.
On admettra des fonctions de transition partielle, c'est-à-dire que la fonction de transition peut ne pas être définie sur certains éléments de son domaine. 
\item
$s \in Q$ est l'état initial de la machine,
\item
$B \in \Gamma - \Sigma$ est le \og \textit{symbole blanc} \fg{} qui occupe initialement toutes les cases vides du ruban.
\item
$F \subseteq Q$ est l'ensemble des états finaux de la machine.
\end{enumerate}

\section{Configuration}

Une \og \textit{configuration} \fg{} d'une machine de Turing contient toute l'information nécessaire à la reprise de l'execution de la machine.
Il s'agit d'un élément de:
\[
Q \times \Gamma^* \times (\{ \epsilon \} \cup \Gamma^* \cdot (\Gamma - \{ B \}))
\]
L'état dans $Q$ désigne l'état de la machine, et les deux autres éléments indiquent le contenu du ruban.
Le premier, dans $\Gamma^*$, indique les symboles à gauche de la tête de lecture.
Le second, dans $\{ \epsilon \} \cup \Gamma^* \cdot (\Gamma - \{ B \}$, indique le contenu dessous et à droite de la tête.
Si ce contenu est uniquement composé de symboles blancs, alors on indique simplement $\epsilon$.
Au cas où ce contenu contient au moins un symbole non-blanc, alors on décrit tous les symboles jusqu'au dernier symbole non-blanc.

À noter que bien que le ruban soit infini, toute configuration ne permet que de décrire des contenus de ruban finis.
Cette restriction au final n'en est pas une, car à tout moment une machine de Turing ne peut avoir consulté et modifié qu'une partie finie du ruban.

\section{Dérivabilité}

Considérons une machine de Turing déterministe $M = (Q, \Gamma, \Sigma, \delta, s, B, F)$ et une configuration $(q, w_L, w_R)$.
Soit $x$ le symbole sous la tête de lecture et $w_R'$ le reste du ruban à droite:
\begin{enumerate}
\item Si $w_R = \epsilon$, dans ce cas $x = B$ et $w_R' = \epsilon$
\item Autrement, $x$ est le premier symbole de $w_R$ et $w_R'$ est le reste.
\end{enumerate}
La fonction de transition $\delta$ retourne pour $(q, x)$ un triplet $(q', x', L)$ ou un triplet $(q', x', L)$:
\begin{enumerate}
\item
Dans le cas $(q', x', L)$, avec comme condition que $w_L$ soit non vide, on obtient une nouvelle configuration $(q', w_L', y \cdot x' \cdot w_R')$, avec $w_L'$ les $|w_L| - 1$ premiers caractères de $w_L$ et $y$ le dernier caractère de $w_L$.
Dans le cas où $w_L$ est vide la transition est impossible.

De plus, pour être exact, nous devons éviter d'introduire des symboles $B$ en fin de mot dans $y \cdot x' \cdot w_R'$.
Dans la configuration, on considérera que tous les $B$ qui apparaissent en fin de mot sont supprimés.

Dans ce cas, on a que:
\[
(q, w_L, w_R) \vdash_M (q', w_L', y \cdot x' \cdot w_R')
\]

\item
Dans le cas $(q', x', R)$, on obtient une configuration $(q', w_L' \cdot x', w_R')$.
On note:
\[
(q, w_L, w_R) \vdash_M (q', w_L' \cdot x', w_R')
\]

\end{enumerate}
On admettra aussi que la fonction de transition soit partielle, c'est-à-dire qu'elle ne retourne rien pour un état et un symbole de ruban donné.
Dans un tel état, on considèrera que la machine ne peut pas entreprendre de transition.

On note $c_1 \vdash_M^* c_2$ pour dénoter le fait qu'une configuration $c_2$ est \og \textit{dérivable} \fg{} d'une configuration $c_1$.
Par définition, $c_1 \vdash_M^* c_2$ si et seulement si il existe un $n \geq 1$ et $n$ configurations $c'_1, \dots, c'_n$ telles que:
\begin{enumerate}
\item $c'_1 = c_1$,
\item $c'_n = c_2$, et
\item Pour tout $1 \geq i < n$, $c'_i \vdash_M c'_{i+1}$.
\end{enumerate}

\section{Langage d'une machine de Turing}

Le langage d'une machine de Turing $M = (Q, \Gamma, \Sigma, \delta, s, B, F)$, noté $L(M)$, est l'ensemble des mots $w$ sur l'alphabet $\Sigma$ tels que:
\[
\exists f, w_1', w_2'.\ (s, \epsilon, w) \vdash_M^* (f, w_1', w_2') \wedge f \in F
\]

\section{Exécution}

Étant donné un mot $w$, une \og \textit{exécution} \fg{} d'une machine de Turing $M = (Q, \Gamma, \Sigma, \delta, s, B, F)$ sur $w$ est une suite de configurations, potentiellement infinie, telle que:
\begin{enumerate}
\item La première configuration est la configuration initiale de la machine:
\[
(s, \epsilon, w)
\]
\item
Chaque configuration $c_{i+1}$ est dérivable en une étape de la configuration précédente $c_i$:
\[
c_i \vdash_M c_{i+1}
\]
\item
La suite est maximale: Il n'existe pas d'exécutions dont elle est strictement le préfixe.
\end{enumerate}
On distingue trois types d'exécution:
\begin{enumerate}
\item Les exécutions finies acceptantes. Dans ce cas, la suite de configurations est finie et la dernière configuration indique un état acceptant dans $F$.
\item Les exécutions finies non-acceptantes. Dans cas, la suite de configurations est finie, et la dernière configuration indique un état non-acceptant. Comme les exécutions sont maximale, on a que la fonction de transition n'est pas définie dans cette dernière configuration.
\item Les exécutions infinies.
\end{enumerate}

\section{Langage accepté}

On dit qu'un langage $L$ est \og \textit{accepté} \fg{} par une machine de Turing $M$ si et seulement si $L(M) = L$.

On parle de langage \og \textit{récursivement énumerable} \fg{} quand un langage est accepté par une machine de Turing.
On note \textit{RE} l'ensemble des langages récursivement énumérables.
Pour chaque langage $L$ dans $RE$, il existe une machine de Turing qui termine et accepte pour chaque mot de $L$.
Pour les mots hors de $L$, la machine est libre de soit terminer et rejeter, soit de boucler.

On note \textit{co-RE} l'ensemble des langages dont le \textit{complément} est dans \textit{RE}.
Pour chaque langage $L$ dans \textit{co-RE}, il existe une machine de Turing qui termine et accepte pour chaque mot qui n'est pas membre de $L$.
Pour les mots de $L$, la machine est libre de soit terminer et rejeter, soit de boucler.

\section{Langage décidé}

On dit qu'un langage $L$ est \og \textit{décidé} \fg{} par une machine de Turing $M$ si et seulement si la machine $M$ accepte $L$ et $M$ n'admet pas pas d'exécutions infinies, peut importe le mot d'entrée.

On parle de langage \og \textit{récursif} \fg{} quand un langage est décidé par une machine de Turing.
On note \textit{R} l'ensemble des langages récursifs. On a que:
\[
R = RE \cap \textit{co-RE}
\]

\section{Thèse de Church-Turing}

La thèse de Church-Turing énonce la correspondance entre problèmes résolubles à l'aide d'une procédure effective et langages décidés à l'aide d'une machine de Turing. Il existe une procédure effective pour résoudre un problème si et seulement si il existe une machine de Turing qui décide le langage des instances positives du problème. Les machines de Turing offrent donc une formalisation de la notion de procédure effective.

Il ne sera cependant pas possible de prouver cette thèse.
En effet, elle fait référence à une notion non formelle de \textit{procédure effective}.
À la place d'une preuve formelle, nous allons argumenter la thèse en montrant que l'expressivité des machines de Turing ne peut pas être amélioré par toutes sortes d'extensions.
De plus, nous explorerons plusieurs autres formalismes d'expressivité équivalente.

\section{Extensions au machines de Turing}

Les variantes suivantes sont des extensions aux machines de Turing.
Étant donnée une machine exprimée à l'aides des extensions suivantes, il sera toujours possible de construire une machine de Turing équivalente mais qui ne fait pas usage de l'extension. 

\subsection{Ruban infini des deux sens}

Les machines de Turing déterministes traditionnelles sont dotées d'un ruban infinie dans une seule direction.
Dans cette version, les cases du ruban sont indexées par les nombres naturels.
Il est possible de lever cette restriction et d'avoir un ruban infini dans les deux sens.
Dans cette version, les cases du ruban sont indexées par les nombres entiers.
Certains cases sont donc d'index négatif.

\subsection{Rubans multiples}

Les machines de Turing déterministes sont dotées d'un seul ruban.
Il est cependant possible d'incorporant de multiples rubans, et ce sans augmenter l'expressivité du formalisme. 

\subsection{Machine de Turing non-déterministes}

Une \og \textit{machine de Turing non-déterministe} \fg{} est une machine de Turing dans laquelle la fonction (partielle) de transition est remplacée par une relation de transition:
\[
\Delta \subseteq ((Q - F) \times \Gamma) \times (Q \times \Gamma \times \{L, R\})
\]

Contrairement au machines de Turing déterministe, une machine de Turing non-déterministe admet de multiples exécutions pour chaque mots d'entrée.
Une mot est accepté par une machine de Turing non-déterministe si et seulement si il existe une exécution finie acceptante.

De manière peut-être contre-intuitive, le non-déterminisme n'augmente pas l'expressivité des machine de Turing.
Étant donnée une machine de Turing non-déterministe $M$ et un mot d'entrée $w$, il est possible de simuler toutes les executions de la machine $M$ sur $w$ à l'aide d'une machine de Turing déterministe.
Pour ce faire, il ne faut pas simuler les exécutions l'une après l'autre, ce qui serait problématique à cause des exécutions infinies, mais simuler les préfixes d'executions pour des tailles de plus en plus grande.
De cette manière, s'il existe une exécution finie acceptante, la machine de Turing déterministe finira par la trouver.
Dans ce cas, il suffit d'accepter le mot d'entrée.
S'il n'existe aucune execution finie acceptante dans $M$, alors la machine de Turing déterministe admettra une execution infinie.

\section{Formalismes équivalents}

\subsection{Lambda calcul}

Le \og \textit{lambda calcul} \fg{} est un langage de programmation fonctionnelle minimaliste.
Le lambda calcul est basé sur la notion d'expression. Les expressions sont des formes suivantes:
\begin{enumerate}
\item Variables, notées $x, y, z$, pour des noms $x, y, z$,
\item Abstractions, notées $\lambda x. \ e$, pour des noms $x$ et expressions $e$,
\item Applications, notées $e_1\ e_2$, pour des expressions $e_1$ et $e_2$.
\end{enumerate}
Le formalisme défini la notion de variables libres et liées, ainsi que celle de substitution, comme nous avons pu les voir dans le contexte du calcul des prédicats.

Une expression du lambda calcul peut être évaluée par un procédé appelé $\beta$-contraction:
Pour toute expression de la forme $(\lambda x.\ e_1)\ e_2$, on peut évaluer l'expression en substituant $e_2$ à $x$ dans $e_1$, c'est-à-dire $e_1[e_2 / x]$, on note:
\[
(\lambda x.\ e_1)\ e_2 \to_\beta e_1[e_2 / x]
\]
En applicant la fermeture iterative de ${\to_\beta}$, on obtient la notion de \og \textit{réduction} \fg{}.
Une expression $e_1$ se \og \textit{réduit} \fg{} à une autre expression $e_2$, noté $e_1 \to_\beta^* e_2$ s'il existe un nombre $n > 1$ et $n$ expressions $e'_1, \dots, e'_2$ tels que:
\begin{enumerate}
\item $e'_1 = e_1$,
\item $e'_n = e_2$, et
\item Pour tout $1 \leq i < n$, $e'_i \to_\beta e'_{i+1}$.
\end{enumerate}

Le lambda calcul est de même expressivité que les machines de Turing.
On dit que le lambda-calcul est \og \textit{Turing-complet} \fg{}.

Le processus d'encodage des instances est cependant différent entre les deux formalismes.
Dans le cas du lambda calcul, l'entrée est encodée comme expression et ensuite passée en argument au programme, qui est lui aussi une expression.
Les \og \textit{encodages de Church} \fg{} sont des techniques d'encodage de certaines valeurs sous forme d'expressions du lambda calcul.

\subsection{Fonctions $\mu$-récursives}

Le livre de référence présente aussi la notion de \og \textit{fonctions $\mu$-récursives} \fg{}, qui sont des fonctions d'entiers à entiers formées uniquement à l'aide de fonctions primitives simples, de composition de fonctions, d'une forme restreinte de récursion, ainsi que d'une opération de minimisation notée $\mu$.

Tout comme le lambda-calcul, le formalisme est Turing-complet:
Pour chaque fonction $\mu$-récursive, il existe une machine de Turing qui simule la fonction.
De même, pour chaque machine de Turing, il existe une fonction $\mu$-récursive qui simule la machine.

À la place d'encoder l'entrée sous forme de mots, une fonction $\mu$-récursive l'encode à l'aide d'un nombre.
Comme l'ensemble des mots sur un alphabet est toujours dénombrable, les deux approches sont équivalentes.

\section{Au delà des problèmes binaires}

Jusqu'à présent, nous avons porté notre attention aux problèmes binaires, c'est-à-dire au problèmes où la réponse était soit \og \textit{oui} \fg{}, soit \og \textit{non} \fg{}.
Les machines de Turing permettent aussi de modéliser des fonctions dont la réponse est plus riche.
Pour ce faire, il suffit de considérer le contenu du ruban à la fin d'une execution comme la sortie de la machine de Turing.

On dit qu'une fonction $f : \Sigma^* \to \Sigma^*$ est \og \textit{calculable} \fg{} par une machine de Turing $M$ si pour tout mot $w \in \Sigma^*$, la machine $M$ a une exécution finie sur $w$ qui termine dans une configuration où $f(w)$ est le contenu du ruban.
Dans ce cas, on ne s'intéressera pas de savoir si l'état est acceptant ou non dans cette dernière configuration.

La thèse de Church-Turing, adaptée à ce contexte, stipule que les fonctions calculables à l'aide d'une procédure effective sont exactement celle qui sont calculable à l'aide d'une machine de Turing.

\section{Machine de Turing universelle}

Il existe une machine de Turing, appelée \og \textit{machine de Turing universelle} \fg{}, qui prend en entrée la représentation d'une machine de Turing quelconque $M$ et un mot d'entrée $w$ de cette machine sous la forme d'une paire $<M, w>$ et qui simule l'exécution de $M$ sur $w$.
Le comportement de la machine de Turing universelle est le suivant:
\begin{itemize}
\item
Si $M$ accepte $w$, alors la machine de Turing universelle accepte $<M, w>$,
\item
Si $M$ termine et rejette $w$, alors la machine de Turing universelle termine et rejette sur $<M, w>$,
\item
Si $M$ boucle sur $w$, alors la machine de Turing universelle boucle elle aussi sur $<M, w>$.
\end{itemize}
La machine de Turing universelle est une sorte d'\textit{interpréteur} de machine de Turing.




% !TEX root = ../cours.tex
\chapter{Indécidabilité}

Dans ce chapitre, nous allons étudier certains problèmes qui sont \og \textit{indécidables} \fg{}, c'est-à-dire des problèmes qui ne sont pas décidés par une machine de Turing.
Pour construire un premier problème indécidable, nous allons à nouveau recourir à la technique de la \textit{diagonalisation}.

\section{Existence d'un langage hors de \textit{RE}}

Rappelons nous que l'ensemble des mots dans $\Sigma^*$ est dénombrable: il existe une bijection $f$ entre $\mathbb{N}$ et $\Sigma^*$.
Notons $f^{-1}$ l'inverse.
De même, l'ensemble des machines de Turing sur un certain alphabet est aussi dénombrable.
Notons $g$ la bijection entre $\mathbb{N}$ et l'ensemble des machines de Turing et $g^{-1}$ son inverse.
Considérons le langage $L$ définit comme:
\[
L_0 \eqdef \{\ w \in \Sigma^*\ |\ g(f^{-1}(w))\text{ n'accepte pas }w\ \}
\]

Admettons, vers une contradiction, que $L_0$ soit dans la classe \textit{RE}.
Il existe donc une machine de Turing $M$ telle que $L(M) = L_0$.
Examinons maintenant l'index $i$ de la machine de Turing $M$ dans la bijection $g$:
\[
i \eqdef g^{-1}(M)
\]
Regardons si $f(i)$, un mot de $\Sigma^*$, appartient à $L_0$.
Considérons les deux cas, soit $f(i) \in L_0$, soit $f(i) \not\in L_0$.
\begin{enumerate}
\item Dans la cas où $f(i) \in L_0$, alors par définition:
\[
g(f^{-1}(f(i))\text{ n'accepte pas }f(i)
\]
Or, $g(f^{-1}(f(i)) = g(i) = M$, et donc $M$ n'accepte pas $f(i)$. On a donc $f(i) \not\in L(M) = L_0$, et donc une contradiction.
\item
Dans le cas où $f(i) \not\in L_0$, alors on a que $f(i) \not\in L(M)$ et donc que $M$ n'accepte pas $f(i)$.
De plus, observons que $M = g(i)$, et $i = f^{-1}(f(i))$. On a donc:
\[
g(f^{-1}(f(i))) \text{ n'accepte pas } f(i)
\]
Par définition, on a donc que:
\[
f(i) \in L_0
\]
Comme $L_0 = L(M)$, on arrive à une contradiction. 
\end{enumerate}

\section{Existence d'un langage dans \textit{RE} mais hors de \textit{co-RE}}

Examinons le complément de $L_0$, noté $\overline{L_0}$:
\[
\overline{L_0} = \{\ w \in \Sigma^*\ |\ g(f^{-1}(w))\text{ accepte }w\ \}
\]
Ce langage fait partie de \textit{RE}: En effet, il est possible de concevoir une machine qui accepte ce langage.
La façon de procédé est la suivante:
\begin{itemize}
\item Étant donné un mot d'entrée $w$, la machine énumère les mots de $\Sigma^*$ jusqu'à trouver l'index $i$ qui correspond à $w$.
\item La machine énumère les machines de Turing jusqu'à la machine $M$ d'index $i$.
\item La machine simule l'exécution de $M$ sur $w$.
\end{itemize}
Le langage $\overline{L_0}$ est donc dans \textit{RE}.
Par contre, il n'est pas dans \textit{co-RE}.
En effet, $L_0$ n'est pas dans \textit{RE}.
On a donc un langage, $\overline{L_0}$, qui est dans \textit{RE} mais pas dans $\textit{R} = \textit{RE} \cap \textit{co-RE}$.
Le langage est \textit{récursivement énumérable}, mais pas \textit{récursif}.
Le problème est dit \textit{indécidable}: il n'existe pas de machine de Turing qui \textit{décide} le langage.

\section{Problème de l'arrêt}

Le problème de l'arrêt consiste à savoir si une machine de Turing s'arrête sur une entrée donnée ou boucle à l'infini. le langage correspond au problème est:
\[
L_\textsc{Halt} \eqdef \{\ <M, w>\ |\ M\text{ s'arrête sur }w\ \}
\]
Comme on va le montrer, le problème de l'arrêt est indécidable.
Pour ce faire, on va procéder par \textit{réduction}: l'idée est de partir de l'hypothèse que le langage est décidable, et de montrer une contradiction en construisant une machine de Turing qui décide un problème indécidable.
Dans le cas présent, on va montrer que s'il existe une machine de Turing $M_\textsc{Halt}$ qui décide de problème de l'arrêt, alors on peut construire une machine de Turing $M_0$ qui décide le langage (indécidable) $\overline{L_0}$.

La construction de la machine de Turing $M_0$ est simple. Il s'agit simplement d'une adaptation de la machine précédemment introduite qui accepte $\overline{L_0}$:
\begin{itemize}
\item Étant donné un mot d'entrée $w$, la machine énumère les mots de $\Sigma^*$ jusqu'à trouver l'index $i$ qui correspond à $w$.
\item La machine énumère les machines de Turing jusqu'à la machine $M$ d'index $i$.
\item La machine utilise $M_\textsc{HALT}$ pour déterminer si $M$ s'arrête ou pas sur $w$. Dans le cas où $M$ bouclerait sur $w$, alors la machine s'arrête immédiatement et rejette le mot.
\item La machine simule l'exécution de $M$ sur $w$. L'exécution est garantie à se point de terminer, car $M_\textsc{HALT}$ accepte $<M, w>$.
\end{itemize}

L'existence d'une telle machine $M_0$ est en contradiction avec le fait que $L_0$ est indécidable.
On en conclut donc que l'hypothèse que le problème de l'arrêt est décidable est fausse.
Le problème de l'arrêt est donc indécidable.

\section{Langage universel}

On appelle le langage universel $LU$ l'ensemble des paires $<M, w>$ telles que $M$ est une machine de Turing qui accepte $w$.
\[
LU \eqdef \{\ <M, w>\ |\ M \text{ accepte } w\ \}
\]
Le langage $LU$ est indécidable.
Il n'existe pas de machine de Turing, qui, étant donné une machine $M$ est un mot $w$, décide si $M$ accepte $w$.
L'indécidabilité de $LU$ peut être montrée simplement par réduction vers $\overline{L_0}$ ou vers le problème de l'arrêt, par exemple. 

% !TEX root = ../cours.tex
\chapter{Complexité}

Dans ce dernier chapitre, nous allons nous intéresser à la notion de \og \textit{complexité} \fg{}.
L'étude de la complexité d'un algorithme cherche à calculer l'utilisation en ressources d'un algorithme ou d'une machine de Turing en fonction de son entrée.
La ressource en question sera généralement le temps.
Dans ce cas, on parle de \textit{complexité temporelle}.
Il est aussi possible d'étudier d'autres métriques que le temps.
Par exemple, la quantité de mémoire requise est souvent intéressante à analyser.
On parle dans ce cas là de \textit{complexité en espace}.
Dans ce chapitre, nous nous concentrerons sur la complexité temporelle.

\section{Notation $O$}

Pour parler de \textit{complexité}, nous allons utiliser la notation $O$ afin d'étudier le comportement asymptotique des fonctions. Comme on le vera par la suite, on modélisera la complexité d'un algorithme comme une fonction de la taille de l'entrée de l'algorithme.

L'ensemble $O(g)$ d'une fonction $g: \mathbb{N} \to \mathbb{N}$ est l'ensemble de toutes les fonctions $f$ telles que $f$ est, à partir d'un certain point, bornée par un multiple de la fonction $g$.

\[
O(g) \eqdef \{\ f\ |\ \exists n_0. \exists c. \forall n.\ n > n_0 \implies f(n) \leq c \cdot g(n)\ \}
\]

Autrement dit, on a que:
\[
f \in O(g) \iff \exists n_0. \exists c. \forall n.\ n > n_0 \implies f(n) \leq c \cdot g(n)
\]

À partir d'un certain point, la fonction $f$ est bornée par la fonction $g$ (multipliée par une constante).

\section{Complexité dans le pire des cas}

La complexité d'une machine de Turing correspond au temps d'exécution de la machine, c'est-à-dire que la complexité correspond au nombre de transitions prises par la machine durant l'exécution.
Évidement, ce nombre dépends généralement de l'entrée.

Décrire exactement le temps d'exécution d'une machine en fonction de l'entrée s'avèrerait trop complexe.
Par simplification, on calcul la complexité en fonction de la \emph{la taille} de l'entrée, et non du contenu exact de l'entrée.
Comme plusieurs entrées partagent la même taille, on considèrera le \emph{pire des cas}.

Au lieu de calculer la complexité de manière exact, on utilisera généralement la notation $O$.
Cette notation permet de s'affranchir de détails d'implémentation.

\section{Classe de complexité $P$}

La classe de complexité $P$ est l'ensemble des langages qui sont décidés par une machine de Turing déterministe dont la complexité dans le pire des cas est bornée par un polynôme.

Intuitivement, la classe $P$ correspond approximativement au problèmes qui sont décidables de façon \emph{efficace}.

\section{Classe de complexité $NP$}

La classe de complexité $NP$ est l'ensemble des langages qui sont décidés par une machine de Turing \textbf{non-déterministe} dont la complexité dans le pire des cas est bornée par un polynôme.

C'est aussi la classe des problèmes où la \textbf{vérification} d'une solution (elle-même de taille polynomiale) au problème par une machine de Turing déterministe prend un temps polynomial.

Déterminer si $P = NP$ est un problème ouvert.
La plupart des chercheurs pensent que $P \neq NP$, mais aucune preuve à ce jour n'existe.

\section{Problèmes $NP$-complets}

Parmi les problèmes $NP$, on considère un sous-ensemble des problèmes \textit{les plus durs}, qu'on appelle problèmes $NP$-complets.
Un problème est $NP$-complet si et seulement si:
\begin{enumerate}
\item Le problème est dans $NP$, et
\item Il est possible de résoudre tout problème dans $NP$ en temps polynomial sur un machine de Turing déterministe étant donné un oracle pour le premier problème.
\end{enumerate} 

\subsection{Théorème de Cook}

Le théorème de Cook énonce que le problème \texttt{SAT} est $NP$-complet.
Le langage du problème \texttt{SAT} consiste en l'ensemble des formules propositionnelles satisfiables.
En général, on restreint le domaine aux formules en forme normale conjonctive: La formules est une conjonctions de clauses, et chaque clause est une disjonctions de littéraux. Chaque littéral est soit une variable, soit la négation d'une variable.

La preuve du théorème montre comment, étant donné une machine de Turing non-déterministe qui s'exécute en un temps polynomial, construire une formule propositionnelle (en temps polynomial) qui soit satisfiable si et seulement si la machine de Turing a une exécution acceptante.

Le théorème de Cook établi l'existence d'un premier problème $NP$-complet.

\subsection{Réduction polynomiales}

Une fois qu'on a établi qu'un problème $A$ est dans NP, pour démontrer que $A$ est NP-complet l'on procède généralement par \og \textit{réduction polynomiale} \fg{}:
On montre qu'il est possible, étant donné un oracle pour le problème $A$, de décider un problème $NP$-complet $B$ en temps polynomial déterministe.

Pour ce faire, on décrit une procédure effective qui prend pour entrée une instance de $B$ et la transforme (en temps polynomial) en instance de $A$.
À ce moment là, l'oracle pour $A$ est appelé et sa réponse est utilisée pour déterminer le résultat.

On établi donc que si $A$ est dans $P$, alors le problème $B$ est lui aussi dans $P$, et par conséquent tous les problèmes dans $NP$ sont dans $P$, et donc $P = NP$. Il suffit donc de trouver un seul algorithme polynomial qui décide d'un seul problème NP-complet afin de montrer que $P = NP$.
Le fait qu'un tel algorithme n'aie pas encore été trouvé fait lourdement penser que $P \neq NP$.


\end{document}
