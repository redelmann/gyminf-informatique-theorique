\documentclass[12pt,french,a4paper]{article}

\usepackage{ae,lmodern}
\usepackage[francais]{babel}
\usepackage[utf8]{inputenc}
\usepackage[T1]{fontenc}
\usepackage{geometry}
 \geometry{
 a4paper,
 total={170mm,257mm},
 left=20mm,
 top=20mm,
 }
\usepackage{exsheets}
\usepackage{amsmath}
\usepackage{amssymb}
\usepackage{mathtools}
\usepackage{proof}
\usepackage{logicpuzzle}
\usepackage{hyperref}
\usepackage{cleveref}

% Définition de la commande pour le signe = avec "déf" aussi dessus.
\newcommand\eqdef{\mathrel{\overset{\makebox[0pt]{\mbox{\normalfont\tiny\sffamily déf}}}{=}}}

\begin{document}

\title{\vspace{-2cm}Série d'exercices n°10\\\large{Fondamentaux formels / Informatique théorique\\GymInf}}
\date{\vspace{-1cm}24 septembre 2021}

\maketitle

\begin{question}
Prouvez les propositions suivantes ou leur négation:
\begin{enumerate}
\item $n \in O(1)$
\item $n \in O(n)$
\item $n \in O(n^2)$
\item $n^2 \in O(n)$
\item $n^2 \in O(2^n)$
\item $2^n \in O(n^2)$
\item $2^n \in O(n!)$
\item $n! \in O(2^n)$
\end{enumerate}
\end{question}

\vspace{4cm}

\begin{question}
Montrez que le problème \texttt{3-SAT}, dont le langage consiste en l'ensemble des formules propositionnelles en forme normale conjonctive avec \textit{au plus 3 littéraux par clause} qui sont satisfiables, est NP-complet. Pour cela, argumentez premièrement que le problème est dans NP.
Dans un deuxième temps, montrez qu'il existe une transformation polynomiale d'un problème NP-complet (comme SAT) à 3-SAT.
\end{question}

\end{document}