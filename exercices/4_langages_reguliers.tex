\documentclass[12pt,french,a4paper]{article}

\usepackage{geometry}
 \geometry{
 a4paper,
 total={170mm,257mm},
 left=20mm,
 top=20mm,
 }
\usepackage[frenchb]{babel}
\usepackage{exsheets}
\usepackage{amsmath}
\usepackage{amssymb}
\usepackage{mathtools}
\usepackage{proof}
\usepackage{logicpuzzle}
\usepackage{hyperref}
\usepackage{cleveref}

% Définition de la commande pour le signe = avec "déf" aussi dessus.
\newcommand\eqdef{\mathrel{\overset{\makebox[0pt]{\mbox{\normalfont\tiny\sffamily déf}}}{=}}}

\begin{document}

\title{\vspace{-2cm}Série d'exercices n°3\\\large{Fondamentaux formels / Informatique théorique\\GymInf}}
\date{\vspace{-1cm}10 août 2021}

\maketitle

\begin{question}

\paragraph{Partie 1}
Prouvez que pour tout nombre naturel $i$, langage $L$ et mot \textit{non-vide} $x$,
si $x \in L^{i + 1}$, alors soit il existe une décomposition de $x$ en un mot \textit{non-vide} $x_1$ de $L$ et un mot $x_2$ de $L^i$, soit $x$ est un mot de $L^i$.

\paragraph{Partie 2}
Prouvez par induction naturelle que tout nombre naturel $i$, langage $L$ et mot \textit{non-vide} $x$,
si $x \in L^{i + 1}$, alors il existe une décomposition de $x$ en un mot \textit{non-vide} $x_1$ de $L$ et un mot $x_2$ de $L^i$.

\paragraph{Partie 3}
Concluez que pour tout langage $L$ et mot \textit{non-vide} $x$,
si $x \in L^*$, alors il existe une décomposition de $x$ en un mot \textit{non-vide} $x_1$ de $L$ et un mot $x_2$ de $L^*$.
Observez que le mot $x_2$ a une taille plus petite que $x$.
\end{question}

\end{document}