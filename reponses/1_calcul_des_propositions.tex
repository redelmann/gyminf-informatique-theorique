\documentclass[12pt,french,a4paper]{article}

\usepackage[T1]{fontenc}
\usepackage[utf8]{inputenc}
\usepackage{geometry}
 \geometry{
 a4paper,
 total={170mm,257mm},
 left=20mm,
 top=20mm,
 }
\usepackage[frenchb]{babel}
\usepackage{exsheets}
\usepackage{amsmath}
\usepackage{amssymb}
\usepackage{mathtools}
\usepackage{proof}
\usepackage{logicpuzzle}
\usepackage{hyperref}
\usepackage{cleveref}

% Définition de la commande pour le signe = avec "déf" aussi dessus.
\newcommand\eqdef{\mathrel{\overset{\makebox[0pt]{\mbox{\normalfont\tiny\sffamily déf}}}{=}}}

\begin{document}


\title{\vspace{-2cm}Réponses à la série d'exercices no1\\\large{Fondamentaux formels / Informatique théorique\\GymInf}}
\date{\vspace{-1cm}9 août 2021}

\maketitle
%************************* Exercice 1
\begin{question}
\begin{enumerate}

\item \begin{displaymath}
\begin{array}{c|ccc}
A  & \neg A & A \vee \neg A\\
\hline
1 & 0 & 1\\
0 & 1 & 1
\end{array} 
\end{displaymath}
et donc $\vDash A \vee \neg A.$

\item Rappel : $A \implies A \implies$ A se lit comme $A \implies (A \implies A).$

Rappel : Table de vérité de $A \implies B$:
\begin{displaymath}
\begin{array}{cc|c}
A & B & A \implies B\\
\hline
1 & 1 & 1\\
1 & 0 & 0\\
0 & 1 & 1\\
0 & 0 & 1
\end{array}
\end{displaymath}

\begin{displaymath}
\begin{array}{c|ccc}
A  & A \implies A & A \implies (A \implies A)\\
\hline
1 & 1 & 1\\
0 & 1 & 1
\end{array} 
\end{displaymath}
et donc $\vDash A \implies A \implies A.$

\item \begin{displaymath}
\begin{array}{c|ccc}
A  & A \implies A & (A \implies A) \implies A\\
\hline
1 & 1 & 1\\
0 & 1 & 0
\end{array} 
\end{displaymath}
et donc $\not\vDash (A \implies A) \implies A.$

\item\begin{displaymath}
\begin{array}{cc|ccc}
A & B & A \implies B & (A \implies B) \implies A &  ((A \implies B) \implies A) \implies A\\
\hline
1 & 1 & 1 & 1 & 1\\
1 & 0 & 0 & 1 & 1\\
0 & 1 & 1 & 0 & 1\\
0 & 0 & 1 & 0 & 1
\end{array}
\end{displaymath}
et donc $\vDash  ((A \implies B) \implies A) \implies A.$ \hfill \emph{(Loi de Peirce)}


\item\begin{displaymath}
\begin{array}{cc|cccc}
A & B & A \wedge B &  \neg A \vee \neg B & \neg(A \wedge B) \iff  \neg A \vee \neg B\\
\hline
1 & 1 & 1 & 0 & 1\\
1 & 0 & 0 & 1 & 1\\
0 & 1 & 0 & 1 & 1\\
0 & 0 & 0 & 1 & 1
\end{array}
\end{displaymath}
et donc $\vDash  \neg(A \wedge B) \iff  \neg A \vee \neg B.$ \hfill \emph{(Loi de De Morgan)}


\item\begin{displaymath}
\begin{array}{cc|cccc}
A & B & A \vee B &  \neg A \wedge \neg B & \neg(A \vee B) \iff  \neg A \wedge \neg B\\
\hline
1 & 1 & 1 & 0 & 1\\
1 & 0 & 1 & 0 & 1\\
0 & 1 & 1 & 0 & 1\\
0 & 0 & 0 & 1 & 1
\end{array}
\end{displaymath}
et donc $\vDash  \neg(A \vee B) \iff  \neg A \wedge \neg B.$ \hfill \emph{(Loi de De Morgan)}

\end{enumerate}
\end{question}

%************************* Exercice 2
\begin{question}
\begin{enumerate}

\item Satisfiable, $\{\ A \mapsto 1, B \mapsto 0\ \} \vDash A \wedge \neg B.$
\item Satisfiable, $\{\ A \mapsto 0\ \} \vDash A \implies \bot.$
\item Satisfiable, $\{ \} \vDash \top.$

\item\begin{displaymath}
\begin{array}{cc|cccc}
A & B & \top \implies A & \bot \wedge B &  A \implies \bot \wedge B &  (\top \implies A) \wedge( A \implies \bot \wedge B) \\
\hline
1 & 1 & 1 & 0 & 0 & 0\\
1 & 0 & 1 & 0 & 0 & 0\\
0 & 1 & 0 & 0 & 1 & 0\\
0 & 0 & 0 & 0 & 1 & 0
\end{array}
\end{displaymath}
Il n'existe aucun modèle, donc insatisfiable.

\paragraph{Remarques}
La négation de la formule est une tautologie, ce que l'on note ici:
\[
\vDash \neg ((\top \implies A) \wedge( A \implies \bot \wedge B))\
\]

À noter que, comme la formule est insatisfiable, la formule est aussi invalide. Il existe une interprétation qui n'est pas un modèle. Pour dénoter l'invalidité, on note:
\[\not \vDash (\top \implies A) \wedge( A \implies \bot \wedge B)\]

\item Satisfiable, $\{\ A \mapsto 1, B \mapsto 0, C \mapsto 1\ \} \vDash (A \vee B \vee C) \wedge (\neg A \vee C) \wedge (\neg C \vee \neg B) \wedge (\neg B \vee \neg A).$

\end{enumerate}
\end{question}


%************************* Exercice 3
\begin{question}
$F = (A \vee B) \wedge \neg (A \wedge B).$

On appelle cette opération le \og \textit{ou exclusif} \fg{}.
\end{question}


%************************* Exercice 4
\begin{question}
\begin{enumerate}
\item
\[
\infer[(\Rightarrow{}I)] {A \implies A} {[A]}
\]

\item
\[
\infer[(\Rightarrow{}I)] {A\wedge B \implies B \wedge A} {
\infer[(\wedge I)] {B \wedge A} {
\infer[(\wedge E \text{ droite})] {B} {[A \wedge B]} & \infer[(\wedge E \text{ gauche})] {A} {[A \wedge B]}
}
}
\]

\newpage
\item
\[
\infer[(\Rightarrow{}I)] {A\wedge (B \wedge C) \implies (A\wedge B) \wedge C} {
\infer[(\wedge I)] {(A \wedge B) \wedge C} {
	\infer[(\wedge I)] {A \wedge B} {
\infer[(\wedge E \text{ gauche})] {A} {[A \wedge (B \wedge C)]} & \infer[(\wedge E \text{ gauche})] {B} {\infer[(\wedge E \text{ droite})] {B \wedge C} {[A \wedge (B \wedge C)]}} } & \infer[(\wedge E \text{ droite})] {C} {\infer[(\wedge E \text{ droite})] {B \wedge C} {[A \wedge (B \wedge C)]}} 
	}
}
\]

\item
\[
\infer[(\Rightarrow{}I)] {A\vee B \implies B \vee A} {
\infer[(\vee E)] {B \vee A}
	{
	[A \vee B]
	& \infer[(\Rightarrow{}I)] {A \implies B \vee A} {\infer[(\vee I \text{ droite})] {B \vee A} {[A]}} 
	& \infer[(\Rightarrow{}I)] {B \implies B \vee A} {\infer[(\vee I \text{ gauche})] {B \vee A} {[B]}} 
	}
}
\]

\end{enumerate}
\end{question}


%************************* Exercice 5
\begin{question}
Nous démontrons le théorème $\vdash A \wedge \neg A \implies \bot$ avec la déduction naturelle donnée en classe. Rappel : $\neg A \eqdef A \implies \bot$, donc $A \wedge \neg A$ est $A \wedge (A \implies \bot)$.

\[
\infer[(\Rightarrow{}I)] {A \wedge (A \implies \bot) \implies \bot} {
\infer[(\Rightarrow{}E)] {\bot} {
\infer[(\wedge E \text{ droite})] {A \implies \bot} {[A \wedge (A \implies \bot)]} & \infer[(\wedge E \text{ gauche})] {A} {[A \wedge (A \implies \bot)]}
}}
\]

\end{question}


%************************* Exercice 6
\begin{question}

On remarque, par définition de la négation, que:
\begin{eqnarray*}
& \neg \neg (A \vee \neg A) \\
\mbox{est} &\neg (A \vee \neg A) \implies \bot \\
\mbox{est} & \big( (A \vee \neg A) \implies \bot\big) \implies \bot \\
\mbox{est} & \Big( \big(A \vee (A \implies \bot) \big) \implies \bot \Big) \implies \bot.
\end{eqnarray*}
Montrons donc $\vdash \Big( \big(A \vee (A \implies \bot) \big) \implies \bot \Big) \implies \bot$.
Ce théorème est montré par l'arbre suivant:

\[
\infer[(\Rightarrow{}I)]{\Big( \big(A \vee (A \implies \bot) \big) \implies \bot \Big) \implies \bot}{
\infer[(\Rightarrow{}E)]{\bot}{[\big(A \vee (A \implies \bot) \big) \implies \bot] & 
\infer[(\vee I \text{ droite})]{A \vee (A \implies \bot)}{
\infer[(\Rightarrow{}I)]{A \implies \bot}{
\infer[(\Rightarrow{}E)]{\bot}{[\big(A \vee (A \implies \bot) \big) \implies \bot] & 
\infer[(\vee I \text{ gauche})]{A \vee (A \implies \bot)}{[A]}
}
}
}
}
}
\]

En logique classique, en utilisant la règle $(\neg \neg E)$, on peut donc montrer le tiers exclu comme théorème:

\[
\infer[(\neg \neg E)]{A \vee (A \implies \bot)}{
\infer[(\Rightarrow{}I)]{\Big( \big(A \vee (A \implies \bot) \big) \implies \bot \Big) \implies \bot}{
\infer[(\Rightarrow{}E)]{\bot}{[\big(A \vee (A \implies \bot) \big) \implies \bot] & 
\infer[(\vee I \text{ droite})]{A \vee (A \implies \bot)}{
\infer[(\Rightarrow{}I)]{A \implies \bot}{
\infer[(\Rightarrow{}E)]{\bot}{[\big(A \vee (A \implies \bot) \big) \implies \bot] & 
\infer[(\vee I \text{ gauche})]{A \vee (A \implies \bot)}{[A]}
}
}
}
}
}
}
\]


\end{question}


\end{document}
