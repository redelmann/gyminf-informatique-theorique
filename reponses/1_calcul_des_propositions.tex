\documentclass[12pt,french,a4paper]{article}

\usepackage{geometry}
 \geometry{
 a4paper,
 total={170mm,257mm},
 left=20mm,
 top=20mm,
 }
\usepackage[frenchb]{babel}
\usepackage{exsheets}
\usepackage{amsmath}
\usepackage{amssymb}
\usepackage{mathtools}
\usepackage{proof}
\usepackage{logicpuzzle}
\usepackage{hyperref}
\usepackage{cleveref}

% Définition de la commande pour le signe = avec "déf" aussi dessus.
\newcommand\eqdef{\mathrel{\overset{\makebox[0pt]{\mbox{\normalfont\tiny\sffamily déf}}}{=}}}

\begin{document}


\title{\vspace{-2cm}Réponses d'exercices no1\\\large{Fondamentaux formels / Informatique théorique\\GymInf}}
\date{\vspace{-1cm}9 août 2021}

\maketitle

\begin{question}
\begin{enumerate}

\item \begin{displaymath}
\begin{array}{c|ccc}
A  & \neg A & A \wedge \neg A\\
\hline
1 & 0 & 1\\
0 & 1 & 1
\end{array} 
\end{displaymath}
et donc $\vDash A \wedge \neg A.$

\item Rappel : $A \implies A \implies A \eqdef A \implies (A \implies A).$

Rappel : $A \implies B \eqdef \neg A \wedge B$. 

\begin{displaymath}
\begin{array}{c|ccc}
A  & \neg A \wedge A & \neg A \wedge (\neg A \wedge A)\\
\hline
1 & 1 & 1\\
0 & 1 & 1
\end{array} 
\end{displaymath}
et donc $\vDash A \implies A \implies A.$

\item \begin{displaymath}
\begin{array}{c|ccc}
A  & \neg A \wedge A & \neg (\neg A \wedge A) \wedge A \\
\hline
1 & 1 & 1\\
0 & 1 & 0
\end{array}
\end{displaymath}
et donc $\not\vDash (A \implies A) \implies A.$

\item\begin{displaymath}
\begin{array}{cc|ccc}
A & B & A \implies B & (A \implies B) \implies A &  ((A \implies B) \implies A) \implies A\\
\hline
1 & 1 & 1 & 1 & 1\\
1 & 0 & 0 & 1 & 1\\
0 & 1 & 1 & 0 & 1\\
0 & 0 & 1 & 0 & 1
\end{array}
\end{displaymath}
et donc $\vDash  ((A \implies B) \implies A) \implies A.$ \hfill \emph{(Loi de Peirce)}


\item\begin{displaymath}
\begin{array}{cc|ccc}
A & B & A \wedge B &  \neg A \vee \neg B\\
\hline
1 & 1 & 1 & 0\\
1 & 0 & 1 & 0\\
0 & 1 & 1 & 0\\
0 & 0 & 0 & 1
\end{array}
\end{displaymath}
et donc $\vDash  \neg(A \wedge B) \iff  \neg A \vee \neg B.$ \hfill \emph{(Loi de De Morgan)}


\item\begin{displaymath}
\begin{array}{cc|ccc}
A & B & A \vee B &  \neg A \wedge \neg B\\
\hline
1 & 1 & 1 & 0\\
1 & 0 & 0 & 1\\
0 & 1 & 0 & 1\\
0 & 0 & 0 & 1
\end{array}
\end{displaymath}
et donc $\vDash  \neg(A \vee B) \iff  \neg A \wedge \neg B.$ \hfill \emph{(Loi de De Morgan)}

\end{enumerate}
\end{question}

\begin{question}
\begin{enumerate}

\item Satisfiable, $(\top, \bot) \vDash A \wedge \neg B.$
\item Satisfiable, $(\top) \vDash A \implies \top.$
\item Satisfiable, $() \vDash \top.$
\item Satisfiable, $(\top) \vDash A \implies \top.$

\item\begin{displaymath}
\begin{array}{cc|cccc}
A & B & \top \implies A & \bot \wedge B &  A \implies \bot \wedge B &  (\top \implies A) \wedge( A \implies \bot \wedge B) \\
\hline
1 & 1 & 1 & 0 & 0 & 0\\
1 & 0 & 1 & 0 & 0 & 0\\
0 & 1 & 0 & 0 & 1 & 0\\
0 & 0 & 0 & 0 & 1 & 0
\end{array}
\end{displaymath}
Il n'existe aucun modèle, donc insatisfiable. $\not \vDash (\top \implies A) \wedge( A \implies \bot \wedge B).$

\item Satisfiable, $(\top, \bot, \top) \vDash (A \vee B \vee C) \wedge (\neg A \vee C) \wedge (\neg C \vee \neg B) \wedge (\neg B \vee \neg A).$

\end{enumerate}
\end{question}

\begin{question}
$F = (A \vee B) \wedge \neg (A \wedge B).$
\end{question}

\end{document}