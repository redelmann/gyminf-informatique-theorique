\documentclass[12pt,french,a4paper]{article}

\usepackage[T1]{fontenc}
\usepackage[utf8]{inputenc}
\usepackage{geometry}
 \geometry{
 a4paper,
 total={170mm,257mm},
 left=20mm,
 top=20mm,
 }
\usepackage[frenchb]{babel}
\usepackage{exsheets}
\usepackage{amsmath}
\usepackage{amssymb}
\usepackage{mathtools}
\usepackage{proof}
\usepackage{logicpuzzle}
\usepackage{hyperref}
\usepackage{cleveref}
\usepackage{xcolor}

% Définition de la commande pour le signe = avec "déf" aussi dessus.
\newcommand\eqdef{\mathrel{\overset{\makebox[0pt]{\mbox{\normalfont\tiny\sffamily déf}}}{=}}}

\begin{document}


\title{\vspace{-2cm}Réponses à la série d'exercices n°2\\\large{Fondamentaux formels / Informatique théorique\\GymInf}}
\date{\vspace{-1cm}10 août 2021}

\maketitle

%****************Exercice 1
\begin{question}
En rouge les variables libres. Les variables liées sont de la même couleur que le quantificateur qui les lie.

\begin{itemize}
\item $f(\textcolor{red}{x}, \textcolor{red}{y})$
\item $P(\textcolor{red}{x}) \wedge P(f(\textcolor{red}{y}, \textcolor{red}{x}))$
\item $\forall y. P(\textcolor{red}{x})$
\item $\textcolor{blue}{\forall} x.\ P(\textcolor{red}{y}, \textcolor{blue}{x}) \vee Q(\textcolor{red}{z})$
\item $\textcolor{blue}{\forall} x. (\textcolor{violet}{\exists} x. P(\textcolor{violet}{x}, \textcolor{violet}{x})) \wedge Q(\textcolor{blue}{x})$
\item $(\textcolor{blue}{\exists} x. P(\textcolor{blue}{x})) \wedge Q(\textcolor{red}{x}) \iff \textcolor{violet}{\forall} y. P(\textcolor{red}{x}, \textcolor{violet}{y})$
\end{itemize}
\end{question}


%****************Exercice 2
\begin{question}
On utilise le prédicat $\texttt{aime}$ d'arité deux. La formule $\texttt{aime}(x, y)$ indique que $x$ aime $y$.
\begin{enumerate}
\item $\forall x. \exists y.\ \texttt{aime}(x, y)$.
\item $\exists y. \forall x.\ \texttt{aime}(x, y)$.
\item $\forall x. (\neg \exists y. \texttt{aime}(y, x)) \implies \forall y. \neg \texttt{aime}(x, y)$
\item $\forall x. (\exists y.\ \texttt{aime}(x, y)) \implies \exists y.\ \texttt{aime}(y, x)$
\item $\forall x. \forall y.\ \texttt{aime}(x, y) \implies \texttt{aime}(y, x)$
\item $(\forall x. \exists y.\ \texttt{aime}(y, x)) \vee \exists x. \forall y.\ \neg \texttt{aime}(y, x)$
\item $\neg (\forall x.\ (\forall y.\ \texttt{aime}(y, x)) \implies \forall y. \texttt{aime}(x, y))$
\end{enumerate}
\end{question}

%****************Exercice 3
\begin{question}
\begin{enumerate}
\item

\[
\infer[(\Rightarrow{}I)]{(\forall x.\ Q(x)) \implies (\exists x.\ Q(x))}{
\infer[(\exists I)]{\exists x.\ Q(x)}{
\infer[(\forall E)]{Q(C)}{[\forall x.\ Q(x)]}
}
}
\]
\item
Considérons l'univers $\{\ 1\ \}$, avec l'interprétation $\mathcal{I}$ telle que:
\begin{align*}
\mathcal{I}(C) &\eqdef 1\\
\mathcal{I}(P) &\eqdef \emptyset\\
\mathcal{I}(Q) &\eqdef \emptyset\\
\mathcal{I}(R) &\eqdef \emptyset
\end{align*}
L'interprétation $\mathcal{I}$ évalue la formule à $0$, et est donc un contre-exemple.
\item La même interprétation $\mathcal{I}$ qu'au point précédent est un contre-exemple.
\item
\[
\infer[(\Rightarrow{}I)]{(\forall x.\ R(x, x)) \implies (\forall y. \exists z.\ R(y, z))}{
\infer[(\forall I)]{\forall y. \exists z.\ R(y, z)}{
\infer[(\exists I)]{\exists z.\ R(y, z)}{
\infer[(\forall E)]{R(y, y)}{[\forall x.\ R(x, x)]}
}
}
}
\]
\item
Considérons l'univers $\{\ 1, 2\ \}$, avec l'interprétation $\mathcal{I}$ telle que:
\begin{align*}
\mathcal{I}(C) &\eqdef 1\\
\mathcal{I}(P) &\eqdef \emptyset\\
\mathcal{I}(Q) &\eqdef \emptyset\\
\mathcal{I}(R) &\eqdef \{\ (1, 1), (2, 2)\ \}
\end{align*}
L'interprétation $\mathcal{I}$ évalue la formule à $0$, et est donc un contre-exemple.
\end{enumerate}
\end{question}

%****************Exercice 4
\begin{question}
\paragraph{Partie 1}
La théorie a pour prédicats unaires $\texttt{homme}$ et $\texttt{mortel}$, et comme constante $\texttt{socrate}$.
Elle admet comme axiomes:
\begin{enumerate}
\item $\forall x.\ \texttt{homme}(x) \implies \texttt{mortel}(x)$
\item $\texttt{homme}(\texttt{socrate})$
\end{enumerate}

\paragraph{Partie 2}

Considérons, par exemple, l'univers $\{ 1, 2 \}$, et une interprétation $\mathcal{I}$ telle que:
\begin{align*}
\mathcal{I}(\texttt{socrate}) &\eqdef 1\\
\mathcal{I}(\texttt{mortel}) &\eqdef \{\ 1, 2\ \}\\
\mathcal{I}(\texttt{homme}) &\eqdef \{\ 1\ \}\\
\end{align*}

\paragraph{Partie 3}

\[
\infer[(\Rightarrow{}E)]{\texttt{mortel}(\texttt{socrate})}{
\infer[(\forall E)]{\texttt{homme}(\texttt{socrate}) \implies \texttt{mortel}(\texttt{socrate})}{
\infer[]{\forall x.\ \texttt{homme}(x) \implies \texttt{mortel}(x)}{}
} & \infer[]{\texttt{homme}(\texttt{socrate})}{}}
\]

\paragraph{Partie 4}

La proposition $\forall x. \texttt{mortel}(x)$ est \textit{indépendante} de la théorie.
\end{question}

\begin{question} 
\end{question}

\begin{question} 
\end{question}


\end{document}
