\documentclass[12pt,french,a4paper]{article}

\usepackage{ae,lmodern}
\usepackage[francais]{babel}
\usepackage[utf8]{inputenc}
\usepackage[T1]{fontenc}
\usepackage{geometry}
 \geometry{
 a4paper,
 total={170mm,257mm},
 left=20mm,
 top=20mm,
 }
\usepackage{exsheets}
\usepackage{amsmath}
\usepackage{amssymb}
\usepackage{mathtools}
\usepackage{proof}
\usepackage{logicpuzzle}
\usepackage{hyperref}
\usepackage{cleveref}

% Définition de la commande pour le signe = avec "déf" aussi dessus.
\newcommand\eqdef{\mathrel{\overset{\makebox[0pt]{\mbox{\normalfont\tiny\sffamily déf}}}{=}}}

\begin{document}


\title{\vspace{-2cm}Réponses à la série d'exercices n°4\\\large{Fondamentaux formels / Informatique théorique\\GymInf}}
\date{\vspace{-1cm}12 août 2021}

\maketitle

%****************Exercice 1
\begin{question}
\begin{enumerate}
\item \texttt{a}
\item \texttt{ab}
\item \texttt{a}, \texttt{b}
\item \texttt{aa}, \texttt{ac}, \texttt{ba}, \texttt{bc}
\item $\epsilon$
\item Aucun mot.
\item Par exemple: \texttt{aa}, \texttt{aaa}, \texttt{bb}
\item Par exemple: \texttt{a}, \texttt{aab}, \texttt{bbaabbbaba}
\item $\epsilon$
\item $\epsilon$. En effet, $\{\}^0 = \{\ \epsilon\ \}$
\end{enumerate}
\end{question}


%****************Exercice 2
\begin{question}
\begin{enumerate}
\item $\texttt{a}\Sigma^{*}$
\item $\texttt{abbb}^{*}\texttt{c}$
\item $\texttt{a}\Sigma^{*}\texttt{c}$
\item $\texttt{a}\Sigma^{*}\texttt{a} \cup \texttt{a}$
\item $\Sigma^{*}\texttt{a}\Sigma^{*}$
\item $\Sigma^{*}(\texttt{a}\cup\texttt{b})\Sigma^{*}$
\item $\Sigma^{*}(\texttt{a}\Sigma^{*}\texttt{b} \cup \texttt{b}\Sigma^{*} \texttt{a})\Sigma^{*}$
\item $(\Sigma\Sigma)^{*}$
\end{enumerate}
\end{question}

%****************Exercice 3
\begin{question}
\begin{enumerate}
\item
Montrons que pour tout $i \in \mathbb{N}$, mot non-vide $x$ et langage $L$, si $x \in L^{i+1}$, alors soit:
\begin{enumerate}
\item
Soit $x$ peut être décomposé en $x_1 \in L$ \textit{non-vide} et $x_2 \in L^i$.
\item
Soit $x \in L^i$. 
\end{enumerate}

Comme $L^{i + 1} = L \cdot L^i$, il existe forcément un mot $x_1 \in L$ et un mot $x_2 \in L^i$ tels que $x = x_1 \cdot x_2$.
On considère deux cas: $x_1 = \epsilon$ et $x_1 \neq \epsilon$:
\begin{enumerate}
\item
Dans le cas $x_1 = \epsilon$, on a que $x = x_1 \cdot x_2 = \epsilon \cdot x_2 = x_2$.
On conclut donc que $x \in L^i$.
\item
Quand $x_1 \neq \epsilon$, alors $x_1$ et $x_2$ est une décomposition de $x$ qui satisfait les conditions.
\end{enumerate}


\item

Montrons que pour tout $i \in \mathbb{N}$, mot non-vide $x$ et langage $L$, si $x \in L^{i+1}$, alors il existe deux mots $x_1$ et $x_2$ tels que:
\begin{enumerate}
\item $x = x_1 \cdot x_2$
\item $x_1 \neq \epsilon$
\item $x_1 \in L$
\item $x_2 \in L^i$
\end{enumerate}
Procédons par induction sur $i$.
\begin{itemize}
\item Considérons le cas où $i = 0$. On a $x \in L^{0+1}$. Observons que $L^{0+1} = L^1 = L$. Posons $x_1 = x$ et $x_2 = \epsilon$. La décomposition $x_1$ et $x_2$ vérifie bien que: 
\begin{enumerate}
\item $x = x_1 \cdot x_2$
\item $x_1 \neq \epsilon$
\item $x_1 \in L$
\item $x_2 \in L^i = L^0 = \{\ \epsilon\ \}$
\end{enumerate}

\item Considérons maintenant le cas inductif, où $i = n + 1$ pour un certain $n$. Notre hypothèse d'induction stipule que:
\[
\forall x'.\ (x' \neq \epsilon \wedge x' \in L^{n+1}) \implies \exists x_1'. \exists x_2'.\ x' = x_1' \cdot x_2' \wedge x_1' \neq \epsilon \wedge x_1' \in L \wedge x_2' \in L^n
\]
Par hypothèse, on a que $x \in L^{(n + 1) + 1}$. Observons $L^{(n + 1) + 1} = L \cdot L^{n + 1}$.
Il existe donc un mot $x_1$ et un mot $x_2$ tels que $x = x_1 \cdot x_2$ avec $x_1 \in L$ et $x_2 \in L^{n + 1}$.
Si $x_1 \neq \epsilon$, alors la preuve conclut immédiatement, avec $x_1$ et $x_2$ comme décomposition.
Considérons donc le cas où $x_1 = \epsilon$. Dans ce cas, on remarque de manière cruciale que $\epsilon \in L$.

Dans ce cas, on observe que $x = x_1 \cdot x_2 = \epsilon \cdot x_2 = x_2$, et donc $x = x_2$. Donc, on a que $x \in L^{n + 1}$.
Par hypothèse d'induction, on a donc qu'il existe un $x_1'$ et un $x_2'$ tels que:
\begin{enumerate}
\item $x = x_1' \cdot x_2'$
\item $x_1' \neq \epsilon$
\item $x_1' \in L$
\item $x_2' \in L^n$
\end{enumerate}
Prouvons que la décomposition $x_1'$ et $x_2'$ est une solution valable.
La seule condition non-triviale est:
\[
x_2' \in L^{n + 1}
\]
Or, comme $\epsilon \in L$, on a que:
\[
\epsilon \cdot x_2' \in L \cdot L^n
\]
Et donc, comme $L \cdot L^n = L^{n + 1}$: 
\[
\epsilon \cdot x_2' \in L^{n + 1}
\]
Et donc:
\[
x_2' \in L^{n + 1}
\]
Ce qui conclut la preuve.
\end{itemize}

\item

Soit $x \in L^*$ un mot non-vide. Par définition, il existe un $i$ tel que $x \in L^i$.
Considérons deux cas: $i = 0$ ou $i = n + 1$ pour un certain $n$.
\begin{enumerate}
\item
Dans le cas où $i = 0$, alors $x \in L^0$ et donc $x = \epsilon$, ce qui est en contradiction avec l'hypothèse que $x$ est non-vide.
\item
Dans le cas où $i = n + 1$ pour un certain $n$, alors le résultat de la partie 2 conclut immédiatement la preuve.
\end{enumerate}
\end{enumerate}
\end{question}

%****************Exercice 4
\begin{question}
Montrons que, pour tout langage régulier $L$:
\begin{align*}
\exists p. p \geq 1 \wedge \forall x.\ x \in L \wedge |x| \geq p \implies \exists x_1. \exists x_2. \exists x_3.\ &x = x_1 \cdot x_2 \cdot x_3\ \wedge\\
&|x_1 \cdot x_2| \leq p\ \wedge\\
&|x_2| \geq 1\ \wedge\\
&\forall k.\ x_1 \cdot x_2^k \cdot x_3 \in L
\end{align*}
Par induction structurelle sur une expression régulière $e$ qui a pour langage $L$:
\begin{enumerate}
\item
Dans le cas où $e = \epsilon$, il suffit de prendre $p = 1$ pour n'avoir aucun mots dans le langage de l'expression de taille au moins $p$, et donc vérifier trivialement la propriété. En effet, $L(\epsilon) = \{\ \epsilon\ \}$, et donc il n'y a aucun mot de taille $1$ ou plus.
\item
Dans le cas où $e = \emptyset$, il suffit de prendre $p = 1$ pour n'avoir aucun mots dans le langage de l'expression de taille au moins $p$, et donc vérifier trivialement la propriété.
\item
Dans le cas où $e = a$ pour un symbole $a \in \Sigma$, il suffit de prendre $p = 2$ pour n'avoir aucun mots dans le langage de l'expression de taille au moins $p$, et donc vérifier trivialement la propriété.
\item
Considérons le cas $e = e_1 \cup e_2$. Dans ce cas, par hypothèses d'induction, il existe un nombre $p_1$ et un nombre $p_2$ qui vérifient la propriété pour $e_1$ et respectivement $e_2$.

Maintenant, il suffit de prendre $p = \text{max}(p_1, p_2)$. En effet, pour tout mot $x \in L(e_1 \cup e_2)$, on a que soit $x \in L(e_1)$, soit $x \in L(e_2)$. Dans le cas où $x \in L(e_1)$, alors par hypothèse d'induction la propriété est vraie pour $x$ ($|x| \geq p$ implique que $|x| \geq p_1$). De manière similaire pour le cas $x \in L(e_2)$.

\item
Considérons le cas $e = e_1 \cdot e_2$. Dans ce cas, par hypothèses d'induction, il existe un nombre $p_1$ et un nombre $p_2$ qui vérifient la propriété pour $e_1$ et respectivement $e_2$.

Maintenant, il suffit de prendre $p = p_1 + p_2 - 1$. En effet, pour tout mot $x \in L(e_1 \cdot e_2)$, on a qu'il existe $x_1 \in L(e_1)$ et $x_2 \in L(e_2)$ avec $x = x_1 \cdot x_2$. Si $|x| \geq p$, alors forcément soit $|x_1| \geq p_1$, soit $|x_2| \geq p_2$. En effet, si $|x_1| < p_1$ et $|x_2| < p_2$, alors $|x_1| \leq p_1 - 1$ et $|x_2| \leq p_2 - 1$, et donc $|x| = |x_1| + |x_2| \leq p_1 - 1 + p_2 - 1 = p - 1$, ce qui est en contradiction avec l'hypothèse que $|x| \geq p = p_1 + p_2 - 1$.
On considère donc deux cas: $|x_1| \geq p_1$, ou $|x_1| \leq p_1 - 1 \wedge |x_2| \geq p_2$:
\begin{enumerate}
\item
Dans le cas $|x_1| \geq p_1$, alors par hypothèse d'induction il existe une décomposition de $x_1$ en $x'_1$, $x'_2$ et $x'_3$ telle que:
\begin{align*}
&|x'_1 \cdot x'_2| \leq p_1\ \wedge\\
&|x'_2| \geq 1\ \wedge\\
&\forall k.\ x'_1 \cdot x_2'^k \cdot x'_3 \in L(e_1)
\end{align*}
La décomposition de $x$ en $x'_1$, $x'_2$ et $x'_3 \cdot x_2$ satisfait toutes les conditions.
\item
Dans le cas $|x_1| \leq p_1 - 1$ et $|x_2| \geq p_2$, alors par hypothèse d'induction il existe une décomposition de $x_2$ en $x'_1$, $x'_2$ et $x'_3$ telle que:
\begin{align*}
&|x'_1 \cdot x'_2| \leq p_2\ \wedge\\
&|x'_2| \geq 1\ \wedge\\
&\forall k.\ x'_1 \cdot x_2'^k \cdot x'_3 \in L(e_2)
\end{align*}
Dans ce cas, la décomposition de $x$ en $x_1 \cdot x'_1$, $x'_2$ et $x'_3$ satisfait toutes les conditions.
\end{enumerate}
\item
Finalement, considérons le cas $e = e_1^*$.
Par hypothèse d'induction, il existe un nombre $p_1$ tel que la propriété est vérifiée sur le langage de $x_1$.
Posons $p = p_1$.
Admettons $x \in L(e^*) = L(e)^*$ tel que $|x| \geq p$. Par conséquent, $x$ est non-vide.
En appliquant le résultat de l'exercice 3 partie 3, alors forcément il existe $x_1 \in L$ et $x_2 \in L^*$ avec $x_1$ non vide et $x = x_1 \cdot x_2$.

On distingue deux cas: Soit $|x_1| < p$, soit $|x_1| \geq p$.
\begin{enumerate}
\item
Dans le cas $|x_1| < p$, la décomposition de $x$ en $\epsilon$, $x_1$ et $x_2$ est une solution qui satisfait toutes les propriétés désirées. En effet:
\begin{align*}
&|\epsilon \cdot x_1| \leq p\ \wedge\\
&|x_1| \geq 1\ \wedge\\
&\forall k.\ \epsilon \cdot x_1^k \cdot x_2 \in L(e_1^*)
\end{align*}
\item
Dans le cas $|x_1| \geq p$, par hypothèse d'induction il existe une décomposition de $x_1$ en $x_1'$, $x_2'$ et $x_3'$ telle que:
\begin{align*}
&|x'_1 \cdot x'_2| \leq p_1\ \wedge\\
&|x'_2| \geq 1\ \wedge\\
&\forall k.\ x'_1 \cdot x_2'^k \cdot x'_3 \in L(e_1)
\end{align*}
La décomposition de $x$ en $x_1'$, $x_2'$ et $x_3' \cdot x_2$ est une solution. En effet:
\begin{align*}
&|x'_1 \cdot x'_2| \leq p\ \wedge\\
&|x'_2| \geq 1\ \wedge\\
&\forall k.\ x'_1 \cdot x_2'^k \cdot (x'_3 \cdot x_2) \in L(e_1^*)
\end{align*}

\end{enumerate}
\end{enumerate}

\end{question}

\end{document}
