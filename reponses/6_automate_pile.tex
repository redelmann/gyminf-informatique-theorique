\documentclass[12pt,french,a4paper]{article}

\usepackage{ae,lmodern}
\usepackage[francais]{babel}
\usepackage[utf8]{inputenc}
\usepackage[T1]{fontenc}
\usepackage{geometry}
\usepackage{dsfont}
 \geometry{
  a4paper,
  total={170mm,257mm},
  left=20mm,
  top=20mm,
}
\usepackage{exsheets}
\usepackage{amsmath}
\usepackage{amssymb}
\usepackage{mathtools}
\usepackage{proof}
\usepackage{hyperref}
\usepackage{cleveref}
\usepackage{tikz}
\usetikzlibrary{automata, arrows.meta, positioning}

\newcommand\eqdef{\mathrel{\overset{\makebox[0pt]{\mbox{\normalfont\tiny\sffamily déf}}}{=}}}



\begin{document}
\title{\vspace{-2cm}Réponses à la série d'exercices n°6\\\large{Fondamentaux formels / Informatique théorique\\GymInf}}
\date{\vspace{-1cm}26 décembre 2021}

\maketitle

%****************Exercice 1
\begin{question}
Les mots acceptés sont 1, 2, et 5.
\end{question}
\begin{question}
Partie 1.

Pour les 4 langages on a : $$\Sigma=\{a,b,c\}$$ $$\Gamma=\{A,B\}$$ $$P=(Q,\Sigma,\Gamma,\Delta,Z,s,F)$$ 
\begin{enumerate}

\item $L_1$
$$Q=\{q_1,q_2,q_3\}$$
$$s=q_1$$
$$F=\{q_3\}$$
$$\Delta=\{((q_1,a,\epsilon),(q_1,A)), ((q_1,\epsilon,\epsilon), (q_2,\epsilon)), ((q_2,b,A),(q_2,\epsilon)), ((q_2,\epsilon,Z),(q_3,Z))\}$$ 

\item $L_2$
$$Q=\{q_1,q_2,q_3,q_4\}$$
$$s=q_1$$
$$F=\{q_4\}$$
$$\Delta=\{((q_1,a,\epsilon),(q_1,A)), ((q_1,\epsilon,\epsilon),(q_2,\epsilon)), ((q_2,b,\epsilon),(q_2,A)),$$
$$((q_2,\epsilon,\epsilon),(q_3,\epsilon)), ((q_3,\epsilon,Z),(q_4,Z)), ((q_3,a,A)(q_3,\epsilon))\}$$ 

\item $L_3$
$$Q=\{q_1,q_2,q_3\}$$
$$s=q_1$$
$$F=\{q_3\}$$
$$\Delta=\{((q_1,a,\epsilon),(q_1,A)), ((q_1,\epsilon,\epsilon),(q_2,\epsilon)), ((q_2,b,A),(q_2,\epsilon)), ((q_2,\epsilon,Z),(q_3,Z)), ((q_3,c,\epsilon),(q_3,\epsilon))\}$$ 

\item $L_4$
$$Q=\{q_1,q_2,q_3,q_4\}$$
$$s=q_1$$
$$F=\{q_4\}$$
$$\Delta=\{((q_1,a,\epsilon),(q_1,\epsilon)), ((q_1,\epsilon,\epsilon),(q_2,\epsilon)), ((q_2,b,\epsilon),(q_2,B)),$$
$$((q_2,\epsilon,\epsilon),(q_3,\epsilon)), ((q_3,c,B),(q_3,\epsilon)), ((q_3,\epsilon,Z)(q_4,Z))\}$$ 

\end{enumerate}
Partie 2.\\
$$L_5 \eqdef \{a^n b^n c^n|n\in \mathds{N}\}$$ 
Admettons que $L_5$ soit régulier.
Il existe donc, d'après le théorème du gonflement hors-contexte, un entier $p$ tel que tout mot $w$ de $L_5$ de taille plus grande ou égale à $p$ peut être décomposé en 5 sous-parties $x, u, y, v, z$ telles que:
\begin{enumerate}
\item $w = x \cdot u \cdot y \cdot v \cdot z$,
\item $|u \cdot v| > 0$
\item $|u \cdot y \cdot v| \leq p$
\item $\forall i, x \cdot u^i \cdot y \cdot v^i \cdot z \in L_5$. 
\end{enumerate}
Considérons le mot $w = a^p b^p c^p$. Clairement $|w| \geq p$ et $w \in L_5$, il existe donc une décomposition de $w$ en $x, u, y, v, z$ qui respecte les propriétés listées.
Considérons deux cas de figures pour les valeurs de $u$ et $v$.
\begin{enumerate}
\item $u$ et $v$ sont chacun formés d'une seule lettre répétée. Il y a donc une lettre de $a, b, c$ qui n'est pas dans $u$ et pas dans $v$.
On a donc que le mot $x \cdot u^2 \cdot y \cdot v^2 \cdot z \in L_5$ et qu'il ne contient pas le même nombre de chaque lettre, ce qui est une contradiction.
\item $u$ ou $v$ est formé d'au moins deux lettres différentes. En répétant ce mot au moins deux fois, on se retrouve donc avec des lettres dans le désordre (un $b$ avant un $a$, ou un $c$ avant un $a$ ou un $b$).
On a donc que le mot $x \cdot u^2 \cdot y \cdot v^2 \cdot z \in L_5$ et qu'il contient des lettres dans le désordre, ce qui est une contradiction.
\end{enumerate}
Comme nous obtenons de toutes manières une contradiction, notre hypothèse de départ est fausse, et donc le langage $L_5$ n'est pas régulier.

\end{question}
\begin{question}
Partie 1.
$$2P=(Q,\Sigma,\Gamma,\Delta,Z,s,F)$$
avec $$\Delta \subseteq (Q\times \Sigma^* \times \Gamma^* \times \Gamma^*)\times(Q\times \Gamma^* \times \Gamma^*)$$
en utilisant les mêmes définitions que dans la page 51 de la théorie.\\

Partie 2.\\
Les configurations de l'automate à deux piles sont des éléments de l'ensemble $Q\times \Sigma^* \times \Gamma^* \times \Gamma^*$\\
Une configuration $(q',w_2,p_1'r_1,p_2'r_2)$ est dérivable en une étape de $(q,w_1 w_2,p_1 r_1,p_2 r_2)$ si et seulement si :
$$(q,w1,p_1,p_2),(q',p_1',p_2')\in \Delta$$
Un mot est accepté par un automate à deux piles si et seulement si il existe deux piles $p_1\in \Gamma^*$ et $p_2\in \Gamma^*$ et un état final $f$ tel que $(s,w,Z,Z)\vdash_P^*(f,\epsilon,p_1,p_2)$\\

Partie 3.\\
$$Q=\{q_1,q_2,q_3,q_4\}$$
$$s=q_1$$
$$F=\{q_4\}$$
$$\Gamma=\{A\}$$
$$\Sigma=\{a,b\}$$

$$\Delta=\{((q_1,a,\epsilon,\epsilon),(q_1,A,A)), ((q_1,\epsilon,\epsilon,\epsilon),(q_2,\epsilon,\epsilon)), ((q_2,b,A,\epsilon),(q_2,\epsilon,\epsilon)),$$
$$((q_2,\epsilon,Z,\epsilon),(q_3,\epsilon,\epsilon)), ((q_3,c,\epsilon,A),(q_3,\epsilon,\epsilon)), ((q_3,\epsilon,Z,Z),(q_4,\epsilon,\epsilon))\}$$ 

\end{question}
\end{document}
